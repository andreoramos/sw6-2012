\section{Target Group}
Our target group is both children with ASD and their guardians. These guardians have certain needs for special tools and gadgets that help to ease the communication between guardians and children.

Five teachers and educators, who work with children with ASD, act as customers. They will provide requirements and information about the institutions' way of working to give us an insight into their daily struggles.

\subsection{Working with Children with ASD}
This section is based upon the statements of a woman with autism \cite{autism.com}, explaining what it is like to live with ASD, and an interview with an educator at Birken, a special kindergarten for children with ASD (see appendix \ref{InterviewMette} for interview notes).

	People with autism are often more visual in their way of thinking. Rather than visualizing thoughts in language and text, they do it in pictures or visual demonstrations. Pictures and symbols are therefore an essential part of the daily tools used by children with ASD and the people interacting with them. Also, children with ASD can have difficulties expressing themselves by writing or talking, and can often more easily use electronic devices to either type a sentence or show pictures, to communicate with people around them.
	Another characteristic with children with ASD is their perception of time. Some of them simply do not understand phrases like "`in a moment"' or "`soon"', they will need some kind of visual indicator that shows how long time they will have to wait.

Different communication tools for children with autism already exist, but many of them rely on a static database of pictures, and often these has to be printed on paper in order to use them as intended. Other tools, such as hour glasses of different sizes and colors, are also essential when working with children with autism, and these tools are either brought around with the child, or a set is kept every place the child might go, being at an institution or at home.

There exists tools today which helps the guardians in their daily life, although -- as stated in Drazenko's quote -- none of them are price-effective enough to be used throughout the institutions. From the quote, it is clear that there is a need for a more cost-effective solution.

\begin{quotation}
The price of the existing solutions are not sufficiently low that we can afford to buy and use them throughout the institute.\\ 
	\begin{flushright}
		- Drazenko Banjak
	\end{flushright}
\end{quotation}