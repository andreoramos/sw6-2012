\section{Target Group}
Autism is a spectrum disorder meaning that one can be little autistic or very autistic \cite{autism.about.com1}. The autism spectrum can vary from people whom are chatty or silent, affectionate or cold, methodical or disorganized. Lisa Jo Rudy states \cite{autism.about.com2} that:''if you've met one person with autism, you've met one person with autism''.

The Autism Society of America defines autism \cite{definitionOfAutism} as follows: ''Autism is a complex developmental disability that typically appears during the first three years of life and is the result of a neurological disorder that affects the normal functioning of the brain, impacting development in the areas of social interaction and communication skills. Both children and adults with autism typically show difficulties in verbal and non-verbal communication, social interactions, and leisure or play activities.

Autism is one of five disorders that falls under the umbrella of Pervasive Developmental Disorders (PDD), a category of neurological disorders characterized by severe and pervasive impairment in several areas of development.''

Children with autistic spectrum disorder are usually very fixated in a single area, meaning that they can become so interested in an area that they completely forget or ignore the world around them. This fixation can be used by educators or teachers to motivate the children to learn, by including this area of interest into the learning process.

   Children with autism may also have difficulties socializing with people, feeling empathy, and understanding other peoples feelings. Lastly children with an autistic spectrum disorder usually wants to structure everything, which is also an important aspect during the planning of their everyday.
   
\subsection{Working with Children with Autism}
People with autism are often more visual in their way of thinking \cite{autism.com}. Rather than thinking in language and text, they think in pictures or visual demonstrations. Pictures and symbols are therefore an essential part of the daily used tool of children with autism and the people interacting with them. Also, children with autism can have a difficult time expressing themselves by writing by hand or talk, and can often more easily use electronic devices, by either typing or showing pictures, to communicate with people around them.

Different communication tools for children with autism already exist, but many of them rely on a static database of pictures, and often these has to be printed on paper in order to use them as intended (see appendix \ref{InterviewMette}). Other tools, such as hour glasses of different sizes and colors, are also essential when working with children with autism, and these tools are either brought around with the child, or a set is kept every place the child might go, being at an institution or at home.
Because this need of many different tool exist, it is relevant to explore the possibilities in using android tablets with customized and customizable software as a tool for children with autism and the people working with them.