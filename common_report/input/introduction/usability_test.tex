\section{Usability Test}
\label{common:sec:usability_test}
%FORM�L
	%Hvad er form�let med en brugevenlighedstest
	%Hvorfor har vi valgt at have det med?
	%Integrationstest, h�nger de forskellige dele sammen for brugeren
	%Individuelle form�l
The system description \ref{common:sec:sys_description} states, that GIRAF is a collection of applications for Android designed for use by children with autism spectrum disorder. %Gentagelse af noget der kommer to afsnit tidligere, kan eventuelt udelades
The usability of GIRAF is important because GIRAF is supposed to be a tool the guardians can use on a daily basis in interaction with children.
GIRAF is built as an alternative to the tools guardians already use when working with children.
Therefore is it important that the guardians want to pick GIRAF instead of the tools they already have.
This requires GIRAF to be a system the guardians can learn to use quickly and a tool the guardians feel comfortable with.\\
\\
The usability test is also a good indicator of how the individual applications are integrated with each other.
As a developer in a development environment the applications you make, may seem like they are integrated very well.
But when introduced to the real world, the user might find that the applications aren't working together at all.

\subsection{Approach}
The most obvious test group is the persons already involved with the project as contacts.
The invitation sent to the test persons can be found in the appendix \ref{}
%FREMGANGSM�DE
		%INVITATION
		%Reference til appendix
	%OPS�TNING
		%Billede og beskrivelse af hvordan det foregik
		%Optagelser, video + lyd - hvorfor?
		%Hvilke opgaver/personer var med til testen?
	%UDF�RSEL
		%Hvordan gik det til?
		%Modtagelse
		%Briefing
		%Test + debriefing
	%EVALUERING
		%Instant data analysis
		
%RESULTAT - IKKE F�LLES