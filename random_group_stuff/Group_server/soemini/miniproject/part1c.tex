Choosing a traditional development approaches works best for a well defined and known problem area, where XP works well for an area which has an exploratory nature. Going full out traditional will not be the best approach for us, while we do have an idea of the scope of the project, we are still in a learning environment, and as such traditional approach will not be optimal. On the other hand, XP is best suited for experienced groups working on a project which may not be fully defined at its inception. Our project is well defined but contains an element of exploration of the solution, as we work with a poorly defined customer group, due to the nature of autism, and as mentioned, we are in a learning environment. On top of this we have a responsibility towards the big group, that what we make need to fit in the big picture and the other groups need to understand, and be able to interact with our project. 

We have chosen to use SCRUM of SCRUMS as the project needs to be further developed, and thus the project needs to be documented thoroughly. Since the major development method is SCRUM, it does make good sense, not to choose XP as our minor group development cycle. This is because we do need to make the documentation for the major group, and choosing XP as a minor development method would still require us to write documentation, which is normally not a major part of the XP approach, thus making SCRUM of SCRUMS more desirable.