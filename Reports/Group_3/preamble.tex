\documentclass[12pt]{report}
\usepackage[ansinew]{inputenc}
\usepackage[T1]{fontenc}
\usepackage{latexsym}
\usepackage{fixltx2e}
\usepackage[lofdepth,lotdepth]{subfig}
\usepackage{graphicx,epstopdf}
\usepackage[absolute]{textpos}
\usepackage[english]{babel}
%\usepackage[latin1]{inputenc}
\usepackage{amsmath,amsfonts}
\usepackage{natbib}
\usepackage{fancyhdr}
\usepackage{wrapfig}
\usepackage{float}
\usepackage[Conny]{fncychap}
\usepackage[usenames,dvipsnames]{color}
\usepackage{color, colortbl}
\usepackage{longtable}
\usepackage{multirow}
\usepackage{algorithmic}
\setcitestyle{numbers,open={[},close={]}}
\usepackage{epsfig}
\usepackage[official,right]{eurosym}
\usepackage{rotating}
\usepackage{hyperref}
\usepackage{rotating}
\hypersetup{pdfborder={0 0 0}}
\usepackage[absolute]{textpos}
\usepackage[rounded]{syntax}
\usepackage{appendix}
\grammarparsep 1pt
\usepackage{xyling}
\usepackage{pdfpages}

\usepackage{slashbox}
\usepackage{verbatim}
\usepackage{float}

\newfloat{Code}{H}{myc}
\allowdisplaybreaks


% Color definitions
\definecolor{light-gray}{gray}{0.95}
% Color definitions


%EPS images snask

%\usepackage{epstopdf}

\newif\ifpdf
\ifx\pdfoutput\undefined
   \pdffalse
\else
   \pdfoutput=1
   \pdftrue
\fi
\ifpdf
   \usepackage{graphicx}
   \usepackage{epstopdf}
   \DeclareGraphicsRule{.eps}{pdf}{.pdf}{`epstopdf #1}
   \pdfcompresslevel=9
\else
   \usepackage{graphicx}
\fi
%eps image snask end
\epstopdfsetup{suffix=}

%semantic udtryk
\usepackage{turnstile}
%$\nststile{Bottom}{Top}$

%\usepackage[tt]{titlepic}


%LOL MARTIN!
%End lool martin

% C# lol?
\usepackage{listings}
% default words comes from lstlang1.sty
\lstset{language=Java,
  basicstyle=\ttfamily\footnotesize\bfseries,
  float,
  columns=flexible,
  morekeywords=[1]{TmdbAPI,TmdbMovie},
  %keywordstyle=[1]\sffamily,
  backgroundcolor=\color{light-gray},
  captionpos=b,
  frame=single,
  breaklines=true, 
  keywordstyle=\color{Blue},
  commentstyle=\color{Green},
  stringstyle=\color{Mahogany},
  showspaces=false,
  showstringspaces=false,
  numbers=left,                   % where to put the line-numbers
  numberstyle=\footnotesize,      % the size of the fonts that are used for the line-numbers
  stepnumber=1
  }  \newenvironment{program}


% Code environment definition --- Java
% Usage 1: \lstinputlisting[style=sw6Java,label=something,caption=Tove]{someFile or someActualCode}   --- is shown in \listoflistings
% Usage 2: \lstinline[style=sw6Java]{someCode}   --- is not shown in \listoflistings
\lstdefinestyle{sw6Java} {
  language=Java,
  basicstyle=\ttfamily\footnotesize\bfseries,
  float,
  columns=flexible,
  morekeywords=[1]{TmdbAPI,TmdbMovie},
  %keywordstyle=[1]\sffamily,
  backgroundcolor=\color{light-gray},
  captionpos=b,
  frame=single,
  breaklines=true, 
  keywordstyle=\color{Black},
  commentstyle=\color{Black},
  stringstyle=\color{Black},
  showspaces=false,
  showstringspaces=false,
  numbers=left,                   % where to put the line-numbers
  numberstyle=\footnotesize,      % the size of the fonts that are used for the line-numbers
  stepnumber=1
}
\lstdefinestyle{sw6} {
  language=Java,
  basicstyle=\ttfamily\footnotesize\bfseries,
  float,
  columns=flexible,
  morekeywords=[1]{TmdbAPI,TmdbMovie},
  %keywordstyle=[1]\sffamily,
  backgroundcolor=\color{light-gray},
  captionpos=b,
  frame=single,
  breaklines=true, 
  keywordstyle=\color{Blue},
  commentstyle=\color{Green},
  stringstyle=\color{Mahogany},
  showspaces=false,
  showstringspaces=false,
  numbers=left,                   % where to put the line-numbers
  numberstyle=\footnotesize,      % the size of the fonts that are used for the line-numbers
  stepnumber=1
}
% Code environment definition --- Java

\usepackage{url}

\definecolor{javared}{rgb}{0.6,0,0} % for strings


\definecolor{javagreen}{rgb}{0.25,0.5,0.35} % comments

\definecolor{javapurple}{rgb}{0.5,0,0.35} % keywords

\definecolor{javadocblue}{rgb}{0.25,0.35,0.75} % javadoc

 
\usepackage{url}%% Define a new 'leo' style for the package that will use a smaller font.
\makeatletter
\def\url@leostyle{%
  \@ifundefined{selectfont}{\def\UrlFont{\sf}}{\def\UrlFont{\small\ttfamily}}}
\makeatother
%% Now actually use the newly defined style.
\urlstyle{leo}


\pagestyle{fancy}
\lhead{}

\newcommand{\code}[1]{\texttt{#1}}
\newcommand{\secref}{section \ref}
\newcommand{\appref}{appendix \ref}
\newcommand{\chapref}{chapter \ref}
\newcommand{\figref}{figure \ref}
\newcommand{\tabelref}{table \ref}
\newcommand{\listref}{listing \ref}
\renewcommand{\headrulewidth}{0.4pt}
\renewcommand{\footrulewidth}{0.4pt}

%Rasmus' kind of lol
\makeatletter
\newenvironment{Figure}{%
\par\addvspace{12pt plus2pt}%
\def\@captype{figure}%
}{%
\par\addvspace{12pt plus2pt}%
}%
\long\def\@makecaption#1#2{%
\vskip\abovecaptionskip
\sbox\@tempboxa{#1: #2}%
\ifdim \wd\@tempboxa >\hsize
#1: #2\par
\else
\global \@minipagefalse
\hb@xt@\hsize{\hfil\box\@tempboxa\hfil}%
\fi
\vskip\belowcaptionskip}
\makeatother
% Rasmus' kind of lol - stop

\setlength{\headheight}{15pt}

%titlepage image halløj
%\usepackage{eso-pic}
%\newcommand\BackgroundPic{
%\put(0,0){
%\parbox[b][\paperheight]{\paperwidth}{%
%\vfill
%\centering
%\includegraphics[width=\paperwidth,height=\paperheight,keepaspectratio]{Images/front-page.png}%
%\vfill
%}}}
%halløj end


%Jesper Stuff

\lstset{
	language=SQL,
  breaklines=true,                                     % line wrapping on
  frame=ltrb,
  framesep=5pt,
  basicstyle=\normalsize,
  keywordstyle=\ttfamily\color{OliveGreen},
  identifierstyle=\ttfamily\color{CadetBlue}\bfseries,
  commentstyle=\color{Brown},
  stringstyle=\ttfamily,
  showstringspaces=false
}

\usepackage[shortlabels]{enumitem}



