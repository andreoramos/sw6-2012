To test the web interface we have made some use cases, which reflects how we assume the system will be used, and we have tested if it works as intended. 

Case 1:
\begin{enumerate}[(a)]
\item Karen Tailor Smith is an educator at Egebakken. She wants to create her own profile in the system. She wants her user name to be KTS and her password to be oliver (the name of her son).

\item She then wants to create a profile for one of the children at Egebakken. The child's name is Eric Carter, she chooses his user name to be ``Eric C'' with the password ``EC''. He should have access to TestAppe1, which is an app on the GIRAF system. She then sets herself as his guardian.

\item Karen realizes that she wrote the wrong phone number for Eric and wants to change it.

\item When the summer comes, Eric is done at Egebakken, so Karen wants to delete his profile. 
\end{enumerate}
Case 2:
\begin{enumerate}[(a)]
\item John Glenn is an educator at Birken, and wants to add a picture to his profile. The picture is of dog and he wants the picture to have following tags: Dog, Animal, and Brown. The tags, Animal and Brown, are not amongst the existing tags. He wants to add these to the tags and add them to his picture.

\item John Glenn if as guardian for a child, and he wants the child be able to see the picture.

\item He also wants to add a new tag, Mammal, to his picture but this is not in tags yet. 

\item He realize that he spelled Mammal wrong so he has to edit it

\item The next day he finds a better picture of a dog and want to delete the old one.
\end{enumerate}
Case 3:
\begin{enumerate}[(a)]
\item Gabriel Ryder is a parent to Mike who attends Birken. Gabriel wants to add a picture of a cat for her son. She wants the picture to have the tags Cat and Small. Small is not in the tags so she has to add it

\item She also want to add the sound of a cat to the picture. She want this sound to have the tags Loud and Cat. Loud is not in the tags so she has to add it.

\item She learns that cats scare Mike, so she want to remove it.
\end{enumerate}

The results of the test cases are shown in \autoref{tab:resultcases}

\begin{table}[H]
	\centering
	\begin{tabular}{|l|l|p{7cm}|}
		\hline
		Case assignment & Result & Note \\ \hline \hline
		1.a & Success & \\ \hline
		1.b & Success &\\ \hline
		1.c & Success&\\ \hline
		1.d & Success &\\ \hline
		2.a & Success &\\ \hline
		2.b & Success &\\ \hline
		2.c & Failed & You are not able to add a tag to an existing picture, a workaround is to delete the picture, then add it again with the desired tags.\\ \hline
		2.d & Success &\\ \hline
		2.e & Success &\\ \hline
		3.a & Success &\\ \hline
		3.b & Success & The link is made, but there is no way to use it in the current version of the web interface.\\ \hline
		3.c & Success &\\ \hline
	\end{tabular}
	\caption{Results of our web interface test cases}
	\label{tab:resultcases}
\end{table}

There were only one of the test cases that did not succeed. The reason is that we have not implemented the feature.