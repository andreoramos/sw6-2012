In our project it was required that all groups used the same development method to keep things simple. This means that the chosen development method may not be perfectly suited for every group, so while we have chosen the Scrum method we have also made some adjustments to fit it smoother for our group.

The key points of the adjustments are as follow

\begin{itemize}
\item The daily scrum meeting
\item The customer involvement
\item The sprint length
\item Incorporating key elements from Xp method
\end{itemize}

We, in our group, have not utilized the daily scrum meeting that is very typical and defining for the scrum method. The reason for this is that we are already sitting together in the same room. We also know each other well because we have worked together in earlier projects. If we felt like we needed to know anything about the other team members progress we could just ask. We simply did not feel the need to waste time every morning by stating the obvious. We see the daily scrum meeting as a tool for communication. We agree that communication is important to make a great product but we have used alternative ways to communicate.

Customer involvement have been a big focus in the multi project. However our group has been in a different situation than the others because our product has been the server which has little to no interest for the customers. Our biggest requirement providers has been the other groups. This has meant that we have not had a release after each sprint that was shown to the customer for feedback. This was only done once and only for the web interface. As mentioned in section [] a usability test was carried out to get a little more feedback from the customers. 

The sprint length has been modified because the ``number of days'' interval would be misleading because we have some days with lectures and others without. Instead we have define the term ``half days'' which covers either from 8.00 to 12.00 or from 12.00 to 16.00. This interval makes it possible to use days, with lectures covering up half the day, as working days. The sprint length has been dynamic and has been decided at each sprint start. A typical sprint length could be 14 half days which means the next 10 half days to occur. This could be a week without any lectures, or two weeks with lectures. 

There is a specific element associated with the Xp method that we felt the need to incorporate into Scrum, being the pair programming. We have incorporated this element because we find it well suited for programming while leaning. Because we had to learn a new programming language, Java servlet, that none of us had ealier experience in, pair programming helped us a lot in the beginning to quickly catch on.