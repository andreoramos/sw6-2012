

\section{Scrum Implementation}
In our project it was required that all groups used the same development method to keep things simple, hence we used Scrum as development method. 

While we have chosen Scrum, some adjustments have been made to better match our personal preferences. 
The key points of the adjustments are as follows:

\begin{description}
\item [Daily Scrum Meetings]
      The daily Scrum meeting, a tool for communicating our daily progress, is a central concept in Scrum. We have chosen not to use this tool, since we work in close quarters and have running dialogue throughout the day, therefore the meeting became redundant.

\item [Customer Involvement]
      Customer involvement has been a big focus in the multi project. However, our group has been in a different situation than the other project groups, as our biggest requirement providers have been the other groups. This meant that we did not have a release after each sprint to show to the customers. However, a mockup of the web interface, seen in \autoref{app:Mock}, was presented to the customers at an early stage in the development, and as mentioned in \autoref{common:sec:usabilitytest}, a usability test was carried out to get feedback. 

\item [Sprint Length]
      The sprint length has been modified because we have some days with lectures and others without. Instead we have defined the term ``half days'' which covers either from 8.00 to 12.00 or from 12.00 to 16.00. This interval makes it possible to use days, with lectures covering up half the day, as working days. The sprint length has been dynamic and has been decided at each sprint start. A typical sprint length could be 10 half days. This could be a week without any lectures, or two weeks with lectures. 

\item [Pair Programming]
      Pair programming has been adopted from the XP development method\cite{larman}. The reason for doing so, is that we find it well suited for programming while learning.

\end{description}







