Our usability test was carried out using the instant data analysis (IDA) method to process our results.

The results are as follow:

\begin{table}
	\centering
	\begin{tabular}{|p{7cm}|l|l|}
		\hline
		Error & Frequency & Category \\
		\hline
		\hline
		Create a profile was difficult and counter intuitive & 3 & 1 \\ \hline
		The home button was only on some sites and confuses the user & 5 & 1 \\ \hline
		/imagenull would magically appear for no reason when a user was created & 3 & 0 \\ \hline
		couple of HTTP errors during the test & 5 & 2 \\ \hline
		Access to apps caused confusion, the user could not figure out what it was used for & 5 & 0 \\ \hline
		Choose file was not further specified and caused confusion & 3 & 0 \\ \hline
		UI size was not filling up the whole screen which some of the persons found inconvenient & 2 & 0 \\ \hline
		QR code missing home path & 3 & 0 \\ \hline
		Add guardian to child was  counter intuitive & 3 & 2 \\ \hline
		Edit profile made the users believe that they had to change password every time & 2 & 1 \\ \hline
		When logging in on child's profile as guardian the dropdown menu was confusing & 4 & 1 \\ \hline
		The user was confused of what to do after login & 1 & 0 \\ \hline
		Edit/Add picture was confusing & 3 & 1 \\ \hline
		The phone number was required but this is not the situation in reality & 5 & 2 \\ \hline
		Missing caption for choosing role causes confusion & 1 & 0 \\ \hline
		The test subject was unsure which tags responded to which checkbox & 2 & 2 \\ \hline
		Confusing, too many options & 1 & 1 \\ \hline
		The test subject tried to login as one of the predefined users because they thought it was their profile & 2 & 0 \\ \hline
		The user lost the overview & 5 & 1 \\ \hline
		The user was not sure which pictures were public when asked to delete a public picture & 1 & 0 \\ \hline
		When deleting pictures, the tags in brackets caused confusion and the text fields implied that you could write in them which is not the case & 1 & 0 \\ \hline
		The test subject was unsure how to get back from the audio page & 1 & 0 \\
		\hline
	\end{tabular}
	\caption{results from usability test}
	\label{tab:results}
\end{table}

11 errors were defined as cosmetic, 7 as serious and 4 as critical. 

The critical errors where that the page broke down with an error which is an indication of programming errors which is clearly a show stopper. another critical error was that you had to include a phone number for a child while it should be optional. This is a requirement that has not been implemented because we have not gotten the request until now.