Our usability test was carried out using the instant data analysis (IDA) method to process our results.

The results are as follow:

\begin{table}
	\centering
	\begin{tabular}{|l|l|l|}
		\hline
		Error & Frequency & Category \\
		\hline
		\hline
		Create a profile was difficult and counter intuitive & 3 & 1 \\
		The home button was only on some sites and confuses the user & 5 & 1 \\ \hline
		/imagenull would magically appear for no reason when a user was created & 3 & 0 \\ \hline
		couple of HTTP errors during the test & 5 & 2 \\ \hline
		Access to apps caused confusion, the user could not figure out what it was used for & 5 & 0 \\ \hline
		?Choose file & 3 & 0 \\ \hline
		?UI size & 2 & 0 \\ \hline
		QR code missing home path & 3 & 0 \\ \hline
		Add guardian to child was  counter intuitive & 3 & 2 \\ \hline
		?Edit profile, password confuses & 2 & 1 \\ \hline
		?Login on child as guardian dropdown & 4 & 1 \\ \hline
		The user was confused of what to do after login & 1 & 0 \\ \hline
		?Edit/Add picture confusion & 3 & 1 \\ \hline
		?No phone number on child & 5 & 2 \\ \hline
		Missing caption for choosing role causes confusion & 1 & 0 \\ \hline
		?Tags checkbox and selected tags & 2 & 2 \\ \hline
		Confusing, too many options & 1 & 1 \\ \hline
		?Login as predefined & 2 & 0 \\ \hline
		The user lost the overview & 5 & 1 \\ \hline
		?Quick access profile & 5 & 0 \\ \hline
		The user was not sure which pictures were public when asked to delete a public picture & 1 & 0 \\ \hline
		?Delete picture tags & 1 & 0 \\ \hline
		?back from sound & 1 & 0 \\
		\hline
	\end{tabular}
	\caption{results from usability test}
	\label{tab:results}
\end{table}

12 errors were listed cosmetic, 7 as serious and 4 as critical.  