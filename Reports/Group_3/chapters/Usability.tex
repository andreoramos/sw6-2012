In this section we have described the tests we have made for our web interface. First we present our results from the usability test, then we present the test cases and the results.

\subsubsection{Usability Test}

Our usability test was carried out using the IDA method, described in \autoref{common:sec:usabilitytest}. The results are as shown in \autoref{tab:results}

\begin{table}[H]
	\scriptsize
	\centering
	\begin{tabular}{|p{7cm}|r|l|}
		\hline
		Error & Frequency & Category \\
		\hline
		\hline
		Create a profile was difficult and counter intuitive & 3 & Serious \\ \hline
		The home button was only on some sites and confuses the user & 5 & Serious \\ \hline
		/imagenull would appear for no reason when a user was created & 3 & Cosmetic \\ \hline
		HTTP errors during the test & 5 & Critical \\ \hline
		Access to apps caused confusion, the user could not figure out what it was used for & 5 & Cosmetic \\ \hline
		Choose file was not further specified and caused confusion & 3 & Cosmetic \\ \hline
		GUI size was not filling up the whole screen which some of the test subjects found inconvenient & 2 & Cosmetic \\ \hline
		QR-code page misunderstanding home link & 3 & Cosmetic \\ \hline
		Add guardian to child was counter intuitive & 3 & Critical \\ \hline
		Edit profile made the users believe that they had to change password every time & 2 & Serious \\ \hline
		When logging in on child's profile as guardian the dropdown menu was confusing & 4 & Serious \\ \hline
		The user was confused of what to do after login & 1 & Cosmetic \\ \hline
		Edit/add picture was confusing & 3 & Serious \\ \hline
		The phone number was required but this is not the situation in reality & 5 & Critical \\ \hline
		Missing caption for choosing role causes confusion & 1 & Cosmetic \\ \hline
		The test subject was unsure which tags corresponded to which check-box & 2 & Critical \\ \hline
		Confusing, too many options & 1 & Serious \\ \hline
		The test subject tried to login as one of the predefined users because they thought it was their profile & 2 & Cosmetic \\ \hline
		The user lost the overview & 5 & Serious \\ \hline
		The user was not sure which pictures were public when asked to delete a public picture & 1 & Cosmetic \\ \hline
		When deleting pictures, the tags in brackets caused confusion and the text fields implied that you could write in them which is not the case & 1 & Cosmetic \\ \hline
		The test subject was unsure how to get back from the audio page & 1 & Cosmetic \\
		\hline
	\end{tabular}
	\caption{Results of the usability test}
	\label{tab:results}
\end{table}

In summation: 11 errors were defined as cosmetic, 7 as serious, and 4 as critical. \\

The critical errors were that the page broke down with an error. another critical error was that a phone number was required for a child, while it should be optional.
This is a requirement that has not been implemented, because we have not received the request before the test.
When prompted to add a guardian to a child there was confusion on how to do that.
The reason is mainly because it is not placed in an intuitive place, and the test subjects had to get help from the test monitor.
The last critical error was the tags selection. The problem was that the user did not know, if a checkbox was assigned to the caption above or below. While this seems as a cosmetic or serious problem at worst, while the user was experimenting with the tags, the system crashed for unknown reasons. We have not been able to recreate this crash so we have rated the problem as critical although the system might not crash.

The rest of the problems found in the test was primarily things missing, bad layout, or misplaced functionality.
Missing things include ``back'' or ``home'' options from certain pages or captions that tells the user what certain fields require as input when filling out formulas. Bad layout is the main reason for confusion when using the system, this could be too much information presented to the user at once.
Misplaced functionality includes the placement of the profile list on the main page, which a lot of users thought related to editing and adding profiles -- leading to confusion. 