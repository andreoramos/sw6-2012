In the \code{server} package there is a total of 10 classes which has been subjected to testing, an overview is shown in \autoref{table:qqOverview}.
The JUnit test cases column shows how many JUnit unit tests has been written for the class.

\begin{table}[H]
  \begin{center}
  \begin{tabular}{l|c|c}
    Class          & JUnit test cases & System test\\
\hline
    Event          & -                &Yes\\
    CommitEvent    & 5                &Yes\\
    RequestEvent   & -                &Yes\\
    EventHandler   & -                &Yes\\
    EventQueue     & 5                &Yes\\
    QueryBuilder   & 20               &Yes\\
    QueryHandler   & -                &Yes\\
    RequestHandler & -                &Yes\\
    CommitHandler  & -                &Yes\\
    XMLBuilder     & -                &Yes\\
  \end{tabular}
  \caption{Queue and query test overview}
  \label{table:qqOverview}
  \end{center}
\end{table}

Unit testing was not found suitable for classes all due the interdependency of other classes in the project. As en example testing the \code{XMLBuilder} with a unit test would require
a \code{ResultSet} object,the easists way to build this is to get it from the \code{QueryHandler}, which in turn require well formed SQL queries from the \code{QueryBuilder}.
It was decided that it would be most efficient to test many of the classes during a system test. JUnit has been used to test the building of SQL queries, singleton implementation,
and that the queue is in fact Fifo. 

The code in \autoref{code:testCase} shows a JUnit test case for \code{QueryBuilder}. It verifies that the \code{QueryBuilder} properly creates an SQL query for inserting an integer value
in the \code{Media} table. Similiar test cases exists for string values and for SQL queries that update or delete rows.
Due to the implementation of \code{Querybuilder} we can cover all values and SQL query types with 9 unit tests. The remaining 11 test cases are for request queries.

\begin{Code}
\begin{lstlisting}[label=code:testCase,language=java,caption=A JUnit test case]
public void testCreateIntValue() throws Exception
{
  String xml = "<sw6ml><Media><Entry action=\"create\">" +
               "<idMedia type=\"int\">1</idMedia>" +
		"</Entry>" +
		"</Media></sw6ml>";
  doc = dom.Dominate(xml);
  ArrayList<String> one = qbuilder.buildQueries(doc);
  assertEquals("INSERT INTO Media values(1);", one.get(0) );	
}
\end{lstlisting}
\end{Code}

All classe in the \code{server} package has one or more corresponding ad hoc testing classes as well. Many of these classes are similiar in nature to the Unit tests.
They are however not automated and mostly rely on \code{System.out.println()} debuggin output requiring a human reader. 
They also include trial and error implementation of unfamiliary technology.
In total these ad hoc testing classes total 614 lines of code


