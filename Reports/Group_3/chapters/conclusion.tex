A system to handle synchronization of data from the various apps and administrate this has been implemented using a combination of Java, Java Servlets, MySQL, HTML, JavaScript and CSS. Although a full implementation has not been achieved, the key functionality has been implemented. 

The apps are not able to retrieve any data form the remote database, but the only thing needed is to implement the code for this, in Oasis. 

The web interface has the possibility for full profile control, but the security issues of being able to have full control without authentication, needs to be handled. Furthermore the usability test has shown various problems in the web interface, shown in \autoref{tab:results}. The settings and stats are not implemented, as we chose to downgrade this part, and focus on the profile control.
 
The database is fully implemented, and meet all requirements stated in \autoref{subsec:databaseReq}

The system is partly implemented, but very open for future work.

%How can we implement a system to handle the synchronization
%of data from the various GIRAF apps and administrate this?
%Here we will conclude on our findings and discoveries throughout this project as well as what our experience of the different parts have been.

%This semester we were introduced to a new way of developing software, through the use of an agile development method. 
%The chosen method was Scrum and the tools new to us and typical for Scrum were: Project Backlog, Sprints, and Scrum meeting.

%The project backlog was very useful for us, once we learned how to use it. Our problem was that we were too unspecific when defining assignments. Once we started thinking more about the assignments and make them more concrete we started benefiting more from the backlog because it provided a great overview of our progress.

%The sprints was also a great success in the beginning of the project. We started out light because we did not had any idea of what we were able to accomplish in one sprint. After the first sprint our estimates became more precise.
%Once we reached the fifth sprint we started to slack on the structure of sprints because we knew what was left to do and did not feel the needs to create sprint backlogs. This was also when we started to prepare the usability test and final release.
%Once we decided that we were done programming we stopped using the sprint backlog altogether. We have not had a release of the system after every sprint but we do not feel that it has affected our result, because we have found alternate ways to test by utilizing usability analysis.

%As mentioned earlier we did not utilize the sprint meeting because we did not feel the need to do so. We do not feel that we have lost anything by negating the meetings because we have had good communication in the group.

%We think we have used the new work method to great success and we have been successful in neglecting tools we did not see the potential in using and implementing other tools that we needed. 

