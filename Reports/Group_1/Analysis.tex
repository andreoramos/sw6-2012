\chapter{Analysis}

To get an greater knowledge on how to work with children with autism and which tool there are in use in the daily work with the children we has been in contaks with  Tove S�by an speech therapy consultant that works with children with autism.\\ 

Based on the meeting with Tove we have learn that one of the tools used in the communication with children with autism is pictogram.
The pictograms is used in a scheme for the day, where all their daily activities are listed in pictograms this giv the children the possibility to go to theirs schema and see what thay are going to next. The pictograms is also used in cummunication with the children the pictograms in used to by the children to make sentences to communicate with other with and later on learn the to speak if this is a problem for the specific child.\\

In the use of pictogram is the problem taht is is very space-occupying and the tool are not very practical to move around. Another problem is when the need for new pictograms the personal that need the new pictogram has they  print them out, cut them out in small squares, and laminate them. So it is also time consuming to produce new pictograms to the children. 

We have also learn that there is various degrees of the autism the child can suffer from, this is not in fokus but as the children can have various degrees of autism this means that the no child are the same, as mention in the introduction,!s�t ref til 1.2 target group!. So this means that you need to be able to adapt the tools to each child.
 
Therefore based on what we have learn from Tove and from our ovn research it would be practical to have a digital version to use pictograms, so rather then have a lot of maps and a lot of pictograms to eachs of the children they would only have to bring one tablet with the samne functionality just digital.

To use these pictogram the children has a "sentencs board" where the children uses papir pictures to create senteces.

