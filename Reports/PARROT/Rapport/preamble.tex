\documentclass[11pt,english,a4paper,twoside,openright]{memoir}
%\documentclass[a4paper,twoside,12pt]{report}

\hyphenation{Help-Desk-Entities} % split words

%
% Our project name
%
\def\projectname{EPIC}
\def\pn{\textit{\projectname{}}}
\title{\pn{}}

%
% Contents = Table of Contents
%
\renewcommand{\contentsname}{Table of Contents}

%
% Packages
%
\usepackage{courier}
\usepackage{natbib}
\usepackage{lmodern}
\usepackage{pifont}
\usepackage{memhfixc}
\usepackage[utf8]{inputenc}
%\usepackage[latin1]{inputenc}}
\usepackage{fix-cm}
\usepackage[T1]{fontenc}
\usepackage[english,danish]{babel}
\usepackage[draft,english]{fixme}
\usepackage{graphicx}
\DisemulatePackage{setspace}
\usepackage{setspace}
\RequirePackage{lineno}
\usepackage{color}
\usepackage{calc}
\usepackage{soul}
\usepackage{textcomp}
\usepackage{pdfpages}
\usepackage{supertabular}
\usepackage{appendix}
\usepackage{varioref}
\usepackage{listings}
%\usepackage{tweaklist}
\usepackage{flafter}
\usepackage{multicol}
\usepackage{multirow}
\usepackage{rotating}
\usepackage{slashbox}
\usepackage{changepage}
\usepackage{longtable}
\usepackage{placeins}
\usepackage[pdftex,
            colorlinks=false,
            pdfduplex=DuplexFlipLongEdge,
            pdfborder={0 0 0},
            pdftitle={\projectname{}},
            pdfauthor={Group s304a,
                3th semester Computer Science,
                AAU,
                Aalborg,
                Denmark},
            pdfsubject={Application Development},
            pdfkeywords={s304a, LaTeX, .NET, C\#, MVC, ASP.NET, MVC 2, \projectname{}, Aalborg, aau},
            plainpages=false,
            bookmarksdepth=subsection
            ]{hyperref}
\usepackage{colortbl}
\usepackage{amsmath}
\usepackage{expdlist}
\usepackage{fourier}
\usepackage{url}
%\usepackage[pdfstartview=FitB]{hyperref}
\usepackage{verbatim}
\usepackage{ifpdf}
\usepackage{tabularx}
\usepackage{here}
\usepackage{float}

\begin{comment}
%
% Homemade copy of tweaklist.sty
%
\def\enumhook{}
\def\enumhooki{}
\def\enumhookii{}
\def\enumhookiii{}
\def\enumhookiv{}
\def\itemhook{}
\def\itemhooki{}
\def\itemhookii{}
\def\itemhookiii{}
\def\itemhookiv{}
\def\descripthook{}
\def\enumerate{%
  \ifnum \@enumdepth >\thr@@\@toodeep\else
    \advance\@enumdepth\@ne
    \edef\@enumctr{enum\romannumeral\the\@enumdepth}%
      \expandafter
      \list
        \csname label\@enumctr\endcsname
        {\usecounter\@enumctr\def\makelabel##1{\hss\llap{##1}}%
          \enumhook \csname enumhook\romannumeral\the\@enumdepth\endcsname}%
  \fi}
\def\itemize{%
  \ifnum \@itemdepth >\thr@@\@toodeep\else
    \advance\@itemdepth\@ne
    \edef\@itemitem{labelitem\romannumeral\the\@itemdepth}%
    \expandafter
    \list
      \csname\@itemitem\endcsname
      {\def\makelabel##1{\hss\llap{##1}}%
        \itemhook \csname itemhook\romannumeral\the\@itemdepth\endcsname}%
  \fi}
\renewenvironment{description}
                 {\list{}{\labelwidth\z@ \itemindent-\leftmargin
                          \let\makelabel\descriptionlabel\descripthook}}
                 {\endlist}
\end{comment}
%
% Set latex to choose png images for pdf, and eps vectors for dvi
%
\ifpdf
\DeclareGraphicsExtensions{.png}
\else
\DeclareGraphicsExtensions{.eps}
\fi

% Add hyphenation package and ignore font warnings (according to
% the official package documentation, they are all safe to ignore)
% This package is needed for breaking words in \texttt
%
\makeatletter
\let\my@@font@warning\@font@warning
\let\@font@warning\@font@info
\makeatother
%\usepackage[htt]{hyphenat}
\usepackage{hyphenat}
\makeatletter
\let\@font@warning\my@@font@warning
\makeatother



%
% Make a newline after paragraphs
%
\makeatletter
\renewcommand\paragraph{
   \@startsection{paragraph}{4}{0mm}
      {-\baselineskip}
      {.5\baselineskip}
      {\normalfont\normalsize\bfseries}}
\makeatother

%
% Make quotation environments italic
%
\makeatletter
\g@addto@macro{\quote}{\itshape}
\g@addto@macro{\quotation}{\itshape}
\makeatother
%
% FiXME package is using the margin as note, and the
% fixme text is appended to the footnote. The exact
% position of the fixme is marked with a indexy
%
\fxsetup{layout={marginclue,footnote,index},innerlayout=}

%
% New burl command... this is not that pretty
%
\newcommand{\burl}[1] {
	\href{#1}{\texttt{#1}}
}
\newcommand{\burlalt}[2] {
	\href{#1}{\texttt{#2}}
}

%%%%%%%%%%%%%%%%%%%%%%%%%%%%%%%%
% Custom Commands
%%%%%%%%%%%%%%%%%%%%%%%%%%%%%%%%

%%%%%%% Top and Tail commands %%%%%%
\newcommand{\myTop}[1]{\vspace{-8mm}  \vspace{0mm} \textit{#1} \vspace{3.4mm} \hrule \vspace{3.4mm} }
%\newcommand{\myTop}[1]{ \vspace{3.4mm} \textit{#1} \  \\  \hrule \  \\}%
\newcommand{\myTail}[1]{ \vspace{3.4mm} \hrule \vspace{3.4mm} \textit{#1} }%

\newcommand{\emptyTop}[0]{\vspace{-6mm}}

%%%%%%%%%%%%%%%%%%%%%%%%%%%%%%%%%%%





%
% Define a new column type
% Tabularx does not support a centered X column,
% so we will create our own
%
\newcolumntype{L}[1]{>{\raggedright\arraybackslash}p{#1}}

%
% Set line spacing
%
\setstretch{1.1}

%
% Set line number margin
%
\setlength{\linenumbersep}{1.5cm}

%
% Set no paragraph indent
%
\setlength{\parindent}{0pt}

%
% Redefine the indent command to make a horizontal space
%
\renewcommand{\indent}{\hspace{0.5cm}}

%
% Set itemize/enumeration/descripthook seperation (this is a small hack using tweaklist package :))
%
\begin{comment}
\renewcommand{\itemhook}{
	\setlength{\itemsep}{0pt}
}
\renewcommand{\descripthook}{
	\setlength{\itemsep}{0pt}
}
\end{comment}
%  
% Enable subfigures in memoir
%
\newsubfloat{figure}
\newsubfloat{table}

%
% Margins
%
\setlrmarginsandblock{3.5cm}{*}{0.75} % Right and left
\setulmarginsandblock{3cm}{*}{1.2}    % Top and bottom

%
% Headings
% - Change normal headers and footers
%
\makeoddhead{headings}{}{}{\small\rightmark}
\makeevenhead{headings}{\small\leftmark}{}{}

\makeoddfoot{headings}{}{}{\small\thepage}
\makeevenfoot{headings}{\small\thepage}{}{}

\makeheadrule{headings}{\textwidth}{\normalrulethickness}
\makefootrule{headings}{\textwidth}{\normalrulethickness}{\footruleskip}

%
% Force LaTeX to not fill the page with extra vertical space
%
\raggedbottom

%
% Section titles
%
\settocdepth{subsection}
\setsecnumdepth{subsubsection}
\maxsecnumdepth{subsubsection}

%
% New chapter style
%
\definecolor{chapterbg}{rgb}{.45,.45,.45}
\makeatletter
\newlength\dlf@normtxtw
\setlength\dlf@normtxtw{\textwidth}
\def\myhelvetfont{\def\sfdefault{mdput}}
\newsavebox{\feline@chapter}
\newcommand\feline@chapter@marker[1][4cm]{%
\sbox\feline@chapter{%
\resizebox{!}{#1}{\fboxsep=1pt%
\colorbox{chapterbg}{\color{white}\bfseries\sffamily\thechapter}%
}}%
\rotatebox{90}{%
\resizebox{%
\heightof{\usebox{\feline@chapter}}+\depthof{\usebox{\feline@chapter}}}%
{!}{\scshape\so\@chapapp}}\quad%
\raisebox{\depthof{\usebox{\feline@chapter}}}{\usebox{\feline@chapter}}%
}
\newcommand\feline@chm[1][4cm]{%
\sbox\feline@chapter{\feline@chapter@marker[#1]}%
\makebox[0pt][l]{% aka \rlap
\makebox[1cm][r]{\usebox\feline@chapter}%
}}
\makechapterstyle{daleif1}{
\renewcommand\chapnamefont{\normalfont\Large\scshape\raggedleft\so}
\renewcommand\chaptitlefont{\normalfont\huge\bfseries\scshape\color{chapterbg}}
\renewcommand\chapternamenum{}
\renewcommand\printchaptername{}
\renewcommand\printchapternum{\null\hfill\feline@chm[2.5cm]\par}
\renewcommand\afterchapternum{\par\vskip\midchapskip}
\renewcommand\printchaptertitle[1]{\chaptitlefont\raggedleft ##1\par}
}
%
% Apply the chapter style
%
\makeatother
\chapterstyle{daleif1}

%
% Figure Captions
%
\captionnamefont{\bfseries\sffamily\small}
\captiontitlefont{\small}
\changecaptionwidth
\captionwidth{.8\textwidth}
\precaption{\vspace{\baselineskip}}


%
% Define colors used in source code boxes
%
\definecolor{red}{rgb}{1,0,0}
\definecolor{green}{rgb}{0,1,0}
\definecolor{yellow}{rgb}{1,1,0}
\definecolor{blue}{rgb}{0,0,1}
\definecolor{darkblue}{rgb}{0,0,0.5}
\definecolor{prggreen}{rgb}{0.18,0.55,0.34}
\definecolor{prgviolet}{rgb}{0.63,0.13,0.94}
\definecolor{prgred}{rgb}{0.65,0.16,0.16}
\definecolor{prgpink}{rgb}{1,0,1}
\definecolor{srcbg}{rgb}{.97,.97,.97}

%
% Source code boxes
%

\lstnewenvironment{source}[2][]{
    \def\lstlistingname{Source Code}
    \lstset{
        aboveskip=0.5cm,
        belowskip=0.5cm,
        linewidth=\textwidth,
        xleftmargin=0.05\textwidth,
        basicstyle=\ttfamily\selectfont\small,
        extendedchars=true,
        columns=flexible,
        escapechar=\$,
        frame=lines,
        numbers=left,
        numberstyle=\tiny,
        numbersep=10pt,
        breaklines,
        tabsize=3,
        breakatwhitespace=true,
        showspaces=false,
        showstringspaces=false,
        language=C,
        prebreak=\Pisymbol{psy}{191},
        emph={[1]root,base,public,private,protected,interface,abstract,class,throws,new,throwable},
        emphstyle={[1]\color{prgred}\bfseries},
        keywordstyle={[1]\bfseries},
        emph={[2]while,for,do,return,goto,if,else,switch,break,default,case},
        emphstyle={[2]\color{prggreen}},
        keywordstyle={[2]\bfseries},
        emph={[3]volatile,long,int,char,string,boolean,unsigned,void,external,double,byte,float,var},
        emphstyle={[3]\color{prggreen}\bfseries},
        keywordstyle={[3]\bfseries},
        morecomment=[l][\color{blue}]{//},
        morecomment=[s][\color{blue}]{/*}{*/},
        morecomment=[s][\color{prgpink}]{"}{"},
        captionpos=b,
        label=#1,
        caption=#2,
    }
}{}

\lstnewenvironment{pseudo}[2][]{
    \def\lstlistingname{Pseudo Code}
    \lstset{
    	aboveskip=0.5cm,
    	belowskip=0.5cm,
        linewidth=0.95\textwidth,
        xleftmargin=0.05\textwidth,
        basicstyle=\ttfamily\selectfont\small,
        extendedchars=true,
        columns=flexible,
        escapechar=\$,
        frame=lines,
        numbers=left,
        numberstyle=\tiny,
        numbersep=10pt,
        breaklines,
        tabsize=3,
        breakatwhitespace=true,
        showspaces=false,
        showstringspaces=false,
        language=C,
        prebreak=\Pisymbol{psy}{191},
        captionpos=b,
        caption={#1},
        label={#2}
    }
}{}

%
% Make a small indent in tables
%
\setlength{\tabcolsep}{10pt}

%
% Fix the page layout
%
\checkandfixthelayout[nearest]



%
% Old chapter style
%
\begin{comment}
\makechapterstyle{coolish}{
    \newif\ifchapternonum
    \renewcommand\printchaptername{}
    \renewcommand\printchapternum{}
    \renewcommand\printchapternonum{\chapternonumtrue}

    \renewcommand\printchaptertitle[1]{
        \reset@font
		\parindent \z@ 
		\vspace*{10\p@}
		\hbox{
		    \vbox{
                \hsize=2cm
		        \begin{tabular}{c}
				    \ifchapternonum
		                \scshape \strut \vphantom{\@chapapp{}} \hphantom{\@chapapp{}} \\
				    \else
				        \scshape \strut \@chapapp{} \\
				    \fi
                    \fbox{
                        \vrule depth 10em width 0pt
                        \vrule height 0pt depth 0pt width 1ex
                                      
                        \ifchapternonum
                              {\LARGE \bfseries \strut \hphantom{\thechapter}}
                        \else
                              {\LARGE \bfseries \strut \thechapter}
                        \fi
                        \vrule height 0pt depth 0pt width 1ex
                    }
                \end{tabular}
            }
            \vbox{
                \advance\hsize by -2cm
                \hrule\par
                \vskip 6pt
                \hspace{1em}
                \Huge \bfseries ##1
            }
        }
        \vskip 60\p@
    }
}
\end{comment}
