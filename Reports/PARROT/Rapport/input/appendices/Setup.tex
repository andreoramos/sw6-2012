\chapter{Setup and Installation}
This section will cover the procedures that a future developer will have to go through in order to get the current version of PARROT up and running.
We will go through how to install the application on a Android tablet, and how to get the PARROT project ready for development in Eclipse \ref{}.\newline

\subsection{Installing the Application}
In order to install the current version of PARROT on an Android tablet, these steps must first be completed:
\begin{itemize}
	\item The tablet must support Android version 3.2, otherwise PARROT will not run.
	\item (Optional) Install Launcher\_004.apk from the GIRAF launcher group on the tablet.
	\item LocalDB.apk from the Admin group must be installed on the tablet.
	\item The folder Pictogram and all that it contains must be copied to /root/Pictogram on the tablet.
\end{itemize}
When all of the above steps are successfully completed, the current version of PARROT.apk can be installed on the tablet.
Installing the launcher is currently optional, but we recommend that it is done, since GIRAF is meant as a packaged project, rather than a number of completely individual applications.

\subsection{Setting up Eclipse for developing PARROT}
While installing the application can give a quick view into how it is structured, the source code needs to be accessed in order to make changes to the application.
In order to help future developers get through this phase without too much trouble, we have written a short list of instructions on how to do this:
\begin{itemize}
	\item Make sure that Eclipse is installed on your computer (Indigo version or newer).
	\item Install the android ADT plug-in for eclipse http://developer.android.com/sdk/eclipse-adt.html\#installing
	\item Make sure that the tools for Android version 3.2 is downloaded.
	\item Go to the tags/project/PARROT folder on SVN and import PARROT and AmbilWarna.
	\item Go to properties>Android for PARROT, and check that the build target is 3.2.
		\subitem While there, make sure that AmbilWarna is included as a library.
	\item Now go to Properties>Java Build Path and make sure that oasislib.jar is there. If not, press \textbf{Add JAR's} and add it from there.

Now the project should be able to compile. Note that in order to run the application, the Android devide or Android Virtual Device needs to be set up according to the description above this.
	
\end{itemize}