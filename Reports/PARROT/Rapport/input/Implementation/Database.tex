\section{Data management}
\subsection{Problem}
In PARROT, we uses a lot of classes specifically desgined for the given situation, such as the Pictogram class, the Category class, and the PARROTProfile class.
These classes contain the exact ammount of information needed in the application, and have been tailored to make it easier for the developpers to understand the flow of the program. However, since all the data used in PARROT is provided by the local database through the Admin functionality, there are some things to take into account.
The primary problem is that the data classes provided by the admin does not match the ones used in PARROT, and therefore needs to be transformed into PARROT objects before they can be used.

\subsection{Solution}
In PARROT, we have solved the problem of data transformation by writing a class called PARROTDataLoader. The purpose of this class is to handle all interaction between the PARROT application and the Admin interface. An object of this class is used whenever it is nescesary to transfer informataion between PARROT and the database. For instance, whenever the application is started, all information about the current user is loaded from the database.

\subsection{Execution}
In order to demonstrate the functionality of PARROTDataLoader, a few pieces of the code will be shown. However, since the class is fairly long, not all will be shown. To get an idea of how the system works, we will show the code for how categories and pictograms are handled.\newline
The first piece of code to be shown is \textbf{loadProfile}.

\begin{verbatim}{}
public PARROTProfile loadProfile(Long childId,Long appId)	
	{
		Profile prof;

		if(childId !=null && appId !=null)
		{
			prof = help.profilesHelper.getProfileById(childId);	//It used to be "currentProfileId"

			Pictogram pic = new Pictogram(prof.getFirstname(), prof.getPicture(), null, null);	//TODO discuss whether this image might be changed
			PARROTProfile parrotUser = new PARROTProfile(prof.getFirstname(), pic);
			parrotUser.setProfileID(prof.getId());
			Setting<String, String, String> specialSettings = app.getSettings();//This object might be null
			if(specialSettings != null)
			{
				//Load the settings
				parrotUser = loadSettings(parrotUser, specialSettings);

				//Add all of the categories to the profile
				int number = 0;
				String categoryString=null;
				while (true)
				{
					//Here we read the pictograms of the categories
					//The settings reader uses this format : category +number | cat_property | value
					try
					{
						categoryString = specialSettings.get("category"+number).get("pictograms");
					}
					catch (NullPointerException e)
					{
						//the value does not exist, so we will not load anymore categories
						break;
					}

					String colourString = specialSettings.get("category"+number).get("colour");
					int col=Integer.valueOf(colourString);
					String iconString = specialSettings.get("category"+number).get("icon");
					String catName = specialSettings.get("category"+number).get("name");
					parrotUser.addCategory(loadCategory(catName,categoryString,col,iconString));
					number++;
				}

				return parrotUser;
			}
			else
			{
				//If no profile is found, return null.
				//It means that the launcher has not provided a profile, either due to an error, or because PARROT has been launched outside of GIRAF.
				return null;
			}
		}
		//If an error has happened, return null
		return null;


	}
\end{verbatim}
First, we check if the ID's have a value different from null. If they do not, an error has happened, and we abort the operation. If they do, we continue and load a profile from the database corresponding to the child currently using PARROT.

\begin{verbatim}{}
	public Category loadCategory(String catName, String pictureIDs,int colour,String iconString)
	{
		Long iconId = Long.valueOf(iconString);
		Category cat = new Category(catName, colour, loadPictogram(iconId));
		ArrayList<Long> listIDs = getIDsFromString(pictureIDs);
		for(int i = 0; i<listIDs.size();i++)
		{
			cat.addPictogram(loadPictogram(listIDs.get(i)));
		}
		return cat;
	}
\end{verbatim}


\begin{verbatim}{}
	public Pictogram loadPictogram(long id)
	{
		Pictogram pic = null;
		Media media=help.mediaHelper.getSingleMediaById(id); //This is the image media //TODO check type

		List<Media> subMedias =	help.mediaHelper.getSubMediaByMedia(media); //TODO find out if this is ok, or if it needs to be an ArrayList
		Media investigatedMedia;
		String soundPath = null;
		String wordPath = null;
		long soundID = -1; //If this value is still -1 when we save a media, it is because the pictogram has no sound.
		long wordID = -1;

		if(subMedias != null)	//Media files can have a link to a sub-media file, check if this one does.
		{
			for(int i = 0;i<subMedias.size();i++) 		
			{
				investigatedMedia =subMedias.get(i);
				if(investigatedMedia.getMType().equals("SOUND"))
				{
					soundPath = investigatedMedia.getMPath();
					soundID= investigatedMedia.getId();
				}
				else if(investigatedMedia.getMType().equals("WORD"))
				{
					wordPath = investigatedMedia.getMPath();
					wordID = investigatedMedia.getId();
				}
			}
		}
		pic = new Pictogram(media.getName(), media.getMPath(), soundPath, wordPath);
		//set the different ID's
		pic.setImageID(id);
		pic.setSoundID(soundID);
		pic.setWordID(wordID);

		return pic;
	}

\end{verbatim}

\subsection{Result}


\subsection{Notes:}


\subsection{Further Reading:}