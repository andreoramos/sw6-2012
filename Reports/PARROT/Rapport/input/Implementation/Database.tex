\section{Data management}
\subsection*{Problem}
In PARROT, we uses a lot of classes specifically designed for any given situation, such as the Pictogram class, the Category class, and the PARROTProfile class.
These classes contain the exact amount of information needed in the application, and have been tailored to make it easier for the developpers to understand the flow of the program.\newline
However, since all the data used in PARROT is provided by the local database through the Admin functionality, there are some things to take into account.
The primary problem is that the data classes provided by the admin does not match the ones used in PARROT, and therefore needs to be transformed into PARROT objects before they can be used.

\subsection*{Solution}
In PARROT, we have solved the problem of data transformation by writing a class called PARROTDataLoader. The purpose of this class is to handle all interaction between the PARROT application and the Admin interface. An object of this class is used whenever it is necessary to transfer information between PARROT and the database. For instance, whenever the application is started, all information about the current user is loaded from the database.

\subsection*{Execution}
To get an idea of how the system works, we will show the code for how categories and pictograms are handled.\newline
The first piece of code to be shown is \textbf{loadProfile} see Source Code \ref{lPf}.

\begin{source}[{lPf}]{The loadProfile method from the PARROTDataLoader}
public PARROTProfile loadProfile(Long childId,Long appId)	
	{
		Profile prof;

		if(childId !=null && appId !=null)
		{
			prof = help.profilesHelper.getProfileById(childId);	//It used to be "currentProfileId"

			Pictogram pic = new Pictogram(prof.getFirstname(), prof.getPicture(), null, null);	//TODO discuss whether this image might be changed
			PARROTProfile parrotUser = new PARROTProfile(prof.getFirstname(), pic);
			parrotUser.setProfileID(prof.getId());
			Setting<String, String, String> specialSettings = app.getSettings();//This object might be null
			if(specialSettings != null)
			{
				//Load the settings
				parrotUser = loadSettings(parrotUser, specialSettings);

				//Add all of the categories to the profile
				int number = 0;
				String categoryString=null;
				while (true)
				{
					//Here we read the pictograms of the categories
					//The settings reader uses this format :
					// category +number | cat_property | value
					try
					{
						categoryString = specialSettings.get("category"+number).get("pictograms");
					}
					catch (NullPointerException e)
					{
						//the value does not exist, so we will not load anymore categories
						break;
					}

					String colourString = specialSettings.get("category"+number).get("colour");
					int col=Integer.valueOf(colourString);
					String iconString = specialSettings.get("category"+number).get("icon");
					String catName = specialSettings.get("category"+number).get("name");
					parrotUser.addCategory(loadCategory(catName,categoryString,
						col,iconString));
					number++;
				}

				return parrotUser;
			}
			else
			{
				//If no profile is found, return null.
				//It means that the launcher has not provided a profile, either due to an error, or because PARROT has been launched outside of GIRAF.
				return null;
			}
		}
		//If an error has happened, return null
		return null;


	}
\end{source}

First, we check if the ID's have a value different from null. If they do not, an error has happened, and we abort the operation.
If they do, we continue and load a profile from the database corresponding to the child currently using PARROT.\newline
Then, we begin the the conversion.
First a Pictogram object is made, which is fed with the first name of the child, as well as the path to the picture of the child.
This pictogram will serve as the child's identity icon.
Then, a PARROTProfile is instantiated with the name of the child, as well as its Pictogram icon.\newline
In order to go on, we load a Setting object from the database. We will not go in depth with this object, as that is the topic of another report (the OASIS report -- not yet published), but suffice to say that it is a wrapper class for a hash table %or is it hash map?
which we use to store data.\newline
The Setting object ``specialSettings'' returned from admin contains information unique to the current child user of the PARROT application, such as visual settings, and pictogram categories.
Now, supposing ``specialSettings'' is not null (in which case we would abort and return null), we load the user specific settings with \textbf{loadSettings}, and goes on to load the categories.\newline
The way we load categories is special, so it will be described in detail.
First of, categories are stored in the specialSettings hash-table in the format: \newline 
\\
$category\_number | category\_property | value$ \newline
\\
So what we do is iterate our way through the ``specialSettings'' object.
We start by trying to get the pictograms corresponding to Category number 0, which will be the first Category in the list.
If the Category exists, we will get a string containing information about the pictograms.
If not, we have already found all the categories (in this case none) belonging to the current user.\newline
After that, we use the same procedure to get the Category's color, icon and category name.
With this information now at hand, we call the \textbf{loadCategory} method, and adds the resulting Category to the PARROTProfile ``parrotUser'', which is the object representing the current user.\newline
After this, we will increment the number, so that we will look for Category number 1.
We will continue to increment the number until we have found all the categories, and after that, we will return the parrotUser object.\newline
\\
Following the description of the loadProfile method, we will describe \textbf{loadCategory} see Source Code \ref{lc}.

\begin{source}[{lc}]{The loadCategory method from the PARROTDataLoader}
	public Category loadCategory(String catName, String pictureIDs,int colour,String iconString)
	{
		Long iconId = Long.valueOf(iconString);
		Category cat = new Category(catName, colour, loadPictogram(iconId));
		ArrayList<Long> listIDs = getIDsFromString(pictureIDs);
		for(int i = 0; i<listIDs.size();i++)
		{
			cat.addPictogram(loadPictogram(listIDs.get(i)));
		}
		return cat;
	}
\end{source}

The purpose of \textbf{loadCategory} is to instantiate a Category object, fill it with pictograms, and return it to the system.
The method starts by converting the ``iconString'' into a Long, which is the ID of a Media object in the database.\newline
After that, we construct a new Category object ``cat'' from the name and color given by inputs of \textbf{loadCategory}, as well as the icon of the Category.
Then comes the part where we transform the string ``pictureIDs'' into a list of numbers.
This is done by a method called \textbf{getIDsFromString}.
The functionality of this method is described with an example:
Suppose a Category is going to contain the items 5, 23, 12, and 18. Then the string pictureIDs will look like this:\newline

5\#23\#12\#18\$\\

The method \textbf{getIDsFromString} will then return the list\newline
\begin{math}
	\{5,23,12,18\}
\end{math}
\\
which corresponds to the IDs.\newline
When we have all the IDs, the method will go through the list, and return the pictogram corresponding to the given ID.
Finally, the Category object ``cat'' is returned.\newline
\\
After having described how we load a Category into the system, we will described how the \textbf{loadPictogram} method works, see Source Code \ref{lp}.

\begin{source}[{lp}]{The loadPictogram method from the PARROTDataLoader}
	public Pictogram loadPictogram(long id)
	{
		Pictogram pic = null;
		Media media=help.mediaHelper.getSingleMediaById(id); //This is the image media //TODO check type

		List<Media> subMedias =	help.mediaHelper.getSubMediaByMedia(media); //TODO find out if this is ok, or if it needs to be an ArrayList
		Media investigatedMedia;
		String soundPath = null;
		String wordPath = null;
		long soundID = -1; //If this value is still -1 when we save a media, it is because the pictogram has no sound.
		long wordID = -1;

		if(subMedias != null)	//Media files can have a link to a sub-media file, check if this one does.
		{
			for(int i = 0;i<subMedias.size();i++) 		
			{
				investigatedMedia =subMedias.get(i);
				if(investigatedMedia.getMType().equals("SOUND"))
				{
					soundPath = investigatedMedia.getMPath();
					soundID= investigatedMedia.getId();
				}
				else if(investigatedMedia.getMType().equals("WORD"))
				{
					wordPath = investigatedMedia.getMPath();
					wordID = investigatedMedia.getId();
				}
			}
		}
		pic = new Pictogram(media.getName(), media.getMPath(), soundPath, wordPath);
		//set the different ID's
		pic.setImageID(id);
		pic.setSoundID(soundID);
		pic.setWordID(wordID);

		return pic;
	}

\end{source}
The responsibility of the method loadPictogram is to return a pictogram, given its input. The input is the ID of a Media object of the type IMAGE.\\
We start by instantiating an empty Pictogram object ``pic''.\newline
Then, we use the admin funtionality given by OasisLib to get the Media object ``media'', which is the image part of the pictogram.
Then, we use the newly loaded object media to get a list of Medias called ``subMedias''.
This list contains the sound and word parts of the pictogram.
Note that this list can be empty.\newline
Now we create 5 new objects, a Media called ``investigatedMedia'', two String objects called ``soundPath'' and ``wordPath'', and Long values ``soundID'' and ``wordID'', which are both set to -1.
If the list ``subMedias'' is not empty, we will iterate our way through it.\\
We set the value of ``investigatedMedia'' to that of the Media object at the current position in ``investigatedMedia'', and prepare to determine what kind of object it is.
We look at ``investigatedMedia'', and determine if its type is SOUND, or if it is WORD.
If the type is SOUND, we set ``soundPath'' to the path of the media, and set ``soundID'' to the ID of the media.
If the type is WORD, we do the same but for ``wordPath'' and ``wordID'' instead.\\
Note that it is possible that the type of ``investigatedMedia'' is a completely different type. In this case, the association is made by one of the other applications, and we simply ignore the media and continue through the list.\\
When we have gone through the list, we initialize ``pic'' as a new Pictogram from the name of media, the path of ``media'', ``wordPath'', and ``soundPath''.
Finally, we set the IDs of ``pic'' to those of the different medias and return ``pic''.\newline
Note that if an ID is -1 at this point, or a path is null, it simply means that the pictogram in question does not contain either a word or a sound.

\subsection*{Result}
The PARROTDataLoader class provides PARROT with objects that can interact with the local database, either to read from the database, or write to it.

\subsection*{Notes}
In the current version of PARROT, the connection to GIRAF launcher is not yet fully made, so instead of loading the Category chosen by the launcher, PARROT instead starts by creating a profile, saving it to the database, and then loading it again.
The functions in the PARROTDataLoader should be able to handle these changes, but will need to be improved to handle more chances of null pointers.\newline
If changes are made to the database, PARROTDataLoader needs to be changed accordingly, so that it will continue to work.\newline
Finally, PARROT does not contain any save functionality in its user interface.
This should be added at a later date.
Also, adding a \textbf{saveProfile} call to PARROT's \textbf{onPause} method is a way of improving data security.

\subsection*{Further Reading}
If a greater understanding of PARROTDataLoader is needed, we suggest reading the Oasis report, as it describes how the database is constructed, which is what the methods found in PARROTDataLoader is based on.