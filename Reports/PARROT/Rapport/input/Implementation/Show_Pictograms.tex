\section{Displaying Pictograms in GridView} %%Name of Feature.
\label{disppict}

\subsection*{Problem}
We need a way import pictograms from the tablet storage drive and displaying them in the application so that they are ready to be used.
Displaying the pictograms also involves showing the associated text peices so that one can easily tell what it represents.
Areas where we will be displaying pictograms are the Talk board and the Management board.

\subsection*{Solution}
Using GridView we can display both the pictograms and its text.
Rather than having the GridView display pictograms, we can display a LinearLayout called ``PictogramView''.
This layout contains an ImageView with a pictogram and a TextView with the associated text.
In that way we are able to show images with text, without having to imprint the text into the image.\\

In order to prepare the pictograms for display, we have to convert them to bitmaps before our Pictogram Adapter class can convert them into views.

\subsection*{Execution}
In order to determine what to show in a GridView, an adapter is needed. Therefore we constructed the Pictogram Adapter in our project, by extending the ``BaseAdapter'' class.
In code segment Source Code \ref{gridviewm} you can see the getView Method, who's function is to return a view for every index in a GridView.
These views are the image and text that you see in the application.\\


\begin{source}[{gridviewm}]{The GridView method}
	public View getView(int position, View convertView, ViewGroup parent) 
	{
		ImageView imageView;
		View view = convertView;
		TextView textView;
		Pictogram pct=cat.getPictogramAtIndex(position);

		LayoutInflater layoutInflater = (LayoutInflater) context.getSystemService(Context.LAYOUT_INFLATER_SERVICE);
		view = layoutInflater.inflate(R.layout.pictogramview, null);

		imageView = (ImageView) view.findViewById(R.id.pictogrambitmap); 
		imageView.setImageBitmap(pct.getBitmap());

		textView = (TextView) view.findViewById(R.id.pictogramtext);
		textView.setTextSize(20);	//TODO this value should be customizable
		if(pct.isEmpty() == false)
		{

			textView.setText(pct.getName());
		}
		else
		{
			textView.setText("");
		}

		view.setPadding(8, 8, 8, 8);
		return view;
	}
\end{source}
We instantiate the different views that we expect to use and start by initializing the Pictogram according to its position in the Category.
The Category, in short, is an object containing both information about itself and a list of pictograms.
This is followed by creating a ``LayoutInflater'' who's job it is to prepare a view for the application.
In this case the pictogramview found through id reference and used to re-initialize the ``View'' we plan on returning.\\
The Imageview and TextView'are supplied with the data they are to show through method calls. 
As long as the Pictogram is not an Empty pictogram, empty being our placeholder for pictograms, the View will be given an Pictogram and its name before being returned.\\

The code fragment Source Code\ref{gbm} is our getBitmap method from the pictogram class.
This method is important for several reasons, it checks for existing bitmaps, making sure that the application do not waste time reconverting image files to bitmaps.
It retrieves information from resources and converts it to bitmaps if the pictogram is supposed to be ``Empty'', or creates the bitmap from the image on the Tablet storage.

\begin{source}[{gbm}]{The getBitmap method fra the Pictogram class}
	public Bitmap getBitmap()
	{
		if(bitmap == null)
		{
			if(isEmpty() == true)
			{
				Resources res = parrent.getResources();
				bitmap = BitmapFactory.decodeResource(res, R.drawable.usynlig);
			}
			else
			{
				bitmap = BitmapFactory.decodeFile(imagePath);
			}
		}
		return bitmap;
	
	}
\end{source}


\subsection*{Result}
The result of this is that we are able to use our adapter to show pictograms in GridViews wherever we need them.
By storing the bitmap, we can change between views much faster than we could otherwise.


\subsection*{Notes}
The input view ``convertView'' in the getView method is an leftover from an earlier iteration of the method. it no longer serves a purpose but ``BaseAdapter'' requires that the function exists with its current inputs.\\

\subsection*{Further Reading}
