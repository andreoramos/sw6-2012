\section{Sentence Board.}
\label{senb} 
\subsection*{Problem:}
For children with a hindrance in verbal communication, pictograms are utilized to help them communicate. 
In the digital version we need some kind of sentence board which has close to the same functionality such pictograms would have if they where used on a physical board. 
Meaning adding, removing, and reorganize the pictograms in the sentence.

\subsection*{Solution:}
We generally use drag and drop for most of the functionality. 
We use it to add, delete, and reorganize the sentence Board. 
We get the pictograms from categories that are shown in a gridview above the sentence board. 
We can drag pictograms from the gridview into the sentence board to add them. 
Likewise, if we are to drag a pictogram out from the sentence board, it will be removed from the sentence board. 
By dragging a pictogram from the sentence board and release the pictogram in the sentence board, it will be shifted to the last position. 
Dragging a pictogram from the gridview and dropping it on top of a pictogram already in the sentence board, it will simply overwrite the old pictogram. 
To help this along, we also fill the sentence board with ``empty'' pictograms, which are there to show that there are currently no pictograms at this specific spaces.

\subsection*{Execution:}
We create a few long click listeners that utilize our BoxDragListener. 
These long click listeners are placed for the gridview we collect pictograms from, and the sentence board itself. 
These listeners are seen in Source Code \ref{sb1}.

%% get code from SpeechBoardFragment.java lines 96 - 106

\begin{source}[{sb1}]{Long clicking a pictogram}
pictogramGrid.setOnItemLongClickListener(new OnItemLongClickListener()
			{

				public boolean onItemLongClick(AdapterView<?> arg0, View view, int position, long id)
				{
					draggedPictogramIndex = position; //TODO make sure that position is the index of the pictogram
					dragOwnerID = R.id.pictogramgrid;
					ClipData data = ClipData.newPlainText("label", "text"); //TODO Dummy. Pictogram information can be placed here instead.
					DragShadowBuilder shadowBuilder = new DragShadowBuilder(view);
					view.startDrag(data, shadowBuilder, view, 0);
					return true;
				}
\end{source}

Here we also draw the shadow, so that the user can see that they are dragging a pictogram.
The rest of the functionality is in the BoxDragListener. See \autoref{dnd}.

When we know that a pictogram is from outside the sentence board and is dropped into the sentence board, we know that we are to add the pictogram to the sentence. At the moment this only occurs when the pictogram are dragged from the gridview.
If the pictogram is dropped on top of an already existing pictogram in the sentence board, it will overwrite the pictogram on the sentence board.\newline
If it is dropped anywhere else in the sentence board, it will add itself to the last entry of the sentence.
In this situation, we need to know where the pictogram is dropped. 
We do this by getting the position of the drop, and then what index the object at this position in the sentence board has. 
This can be seen in Source Code \ref{sb2}.

%% get code from BoxDragListener lined 54 - 59
\begin{source}[{sb2}]{Dropping a view into the sentence board}
if( self.getId() == R.id.sentenceboard && SpeechBoardFragment.dragOwnerID != R.id.sentenceboard)	//We are about to drop a view into the sentenceboard
				{
					GridView speech = (GridView) parrent.findViewById(R.id.sentenceboard);
					int x = (int)event.getX();
					int y = (int)event.getY();
					int index = speech.pointToPosition(x, y);
\end{source}
We then check if there already is a pictogram in the given place. We can do this by checking if the current pictogram is the ``empty'' pictogram. If it is not, we know that there already is a pictogram, and we replace it.
This is seen in Source Code \ref{sb3}.

%% get code from BoxDragListener lines 71 - 75
\begin{source}[{sb3}]{Replacing a pictogram in the sentence board}
if(SpeechBoardFragment.speechBoardCategory.getPictogramAtIndex
	(index).isEmpty() == false) //Replaces a pictogram already in the sentencebord
						{
							//Removes the pictogram at the specific index
							SpeechBoardFragment.speechBoardCategory.removePictogram(index); 
								
							SpeechBoardFragment.speechBoardCategory.addPictogramAtIndex
								(draggedPictogram, index); //add the pictogram at the specific position
						}
\end{source}
If it is the ``empty'' pictogram, it will try to place the pictogram furthest to the left of the sentence. 
We do this with a while loop that runs for the duration of spaces in the sentence board. We start at the leftmost entry in the sentence board if its empty, and continue to the right. 
When an empty place is found, the pictogram is placed here and the loop breaks.
This is shown in Source Code \ref{sb4}.

%% get code from BoxDragListener.java lines 78 - 90
\begin{source}[{sb4}]{Finding an empty space}
else 
	{
	int count = 0;
	while (count < numberOfSentencePictograms) 
		{

		if (SpeechBoardFragment.speechBoardCategory.getPictogramAtIndex(count).isEmpty() == true) 
			{
			SpeechBoardFragment.speechBoardCategory.removePictogram(count); //Removes the pictogram at the specific index
			SpeechBoardFragment.speechBoardCategory.addPictogramAtIndex(draggedPictogram, count); //add the pictogram at the specific position
			break;
			} 
			count++;
		}
	}
\end{source}
When we know that a pictogram is from the sentence board and is dropped into the sentence board, it is because we are about to rearrange the pictograms in the sentence board. We use the same method as we did when we where about to drag something from the gridview into the sentence board. 

When we know that a pictogram is from the sentence board and dropped somewhere else, we simply remove it from the sentence board. This is already done in the DRAG\_STARTED action. Whenever a pictogram is being dragged from the sentence board, it is removed from it. It can then be placed again via the DROP action. If this happens outside the sentence board though, it is simply removed. 
This method removes the pictogram, and shifts all pictograms that are to the right of it, one to the left, so that no spaces are left in the sentence. 
The deletion of a pictogram is seen in Source Code \ref{sb5}.

% get code from BoxDragListener.java lines 29- 45
\begin{source}[{sb5}]{Adding empty pictograms to the sentence board when pictograms are removed.}
if (event.getAction() == DragEvent.ACTION_DRAG_STARTED){
			if(self.getId() == R.id.sentenceboard && SpeechBoardFragment.dragOwnerID == R.id.sentenceboard)
			{
				draggedPictogram = SpeechBoardFragment.speechBoardCategory.getPictogramAtIndex
					(SpeechBoardFragment.draggedPictogramIndex);
					
				if(draggedPictogram.isEmpty()==true)
				{
					//Do not allow dragging empty pictograms
				}
				else
				{
					GridView speech = (GridView) parrent.findViewById(R.id.sentenceboard);
					
					SpeechBoardFragment.speechBoardCategory.removePictogram
						(SpeechBoardFragment.draggedPictogramIndex);
							
					SpeechBoardFragment.speechBoardCategory.addPictogram(new Pictogram("#usynlig#", null, null, null, parrent));
					
					speech.setAdapter(new PictogramAdapter(SpeechBoardFragment.speechBoardCategory, parrent));
				}
			}
\end{source}

\subsection*{Result:}
We can now add and remove pictograms from the sentence board. We can replace pictograms already on the board. The pictograms will always be shifted so that there are no spaces between picograms. It is possible to rearrange the pictograms.


\subsection*{Notes:}
It should be noted that we have chosen that the pictograms shift to the left whenever they can. This is a design choice that can be changed if further study shows it to be favorable to do otherwise. 
In the same way we have made the choice that pictograms are to be overwritten. Another alternative would be to shift all pictograms to make room for the new pictogram. This is also something that can be changed if further study shows that it is necessary.


\subsection*{Further Reading:}
These functionalities are closely bound to the Drag And Drop functionality see \autoref{dnd}.