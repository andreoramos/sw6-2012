\section{Colour Picker}%Come up with a better name
\subsection{Problem}
Since no two children with autism are the same, our application needs to be customizable. To represent this, we want our application to enable the user to configure the colours of different parts of the application.

\subsection{Solution}
To this problem, the solution is fairly simple. As the WOMBAT group have already found a solution to the problem, we are free to use that solution through data sharing. %is that what we called it, or was it something else?
They have shown they use an Android library project called AmbilWarna in their own project, which has been a great help to implementing it ourselves. The AmbilWarna library provides the application with a dialog box that allows the user to pick a colour as an input to the system.

\subsection{Execution}{}
The use of the AmbilWarna dialog is best seen in the optionsfragment, more precisely when the \textbf{Change Category Colour} button is clicked.

\begin{source}{}
		ccc.setOnClickListener(new OnClickListener() {
			public void onClick(View v) {
				AmbilWarnaDialog dialog = new AmbilWarnaDialog(getActivity(),
						PARROTActivity.getUser().getCategoryColor(), new OnAmbilWarnaListener() {
					public void onCancel(AmbilWarnaDialog dialog) {
					}

					public void onOk(AmbilWarnaDialog dialog, int color) {
						PARROTProfile user = PARROTActivity.getUser();
						user.setCategoryColor(color);
						PARROTActivity.setUser(user);
					}
				});
				dialog.show();
			}
		});
\end{source}

As seen in the code, if the button is clicked, a new AmbilWarna dialog will pop up. The default selected colour of the dialog will be 
\begin{source}{}
PARROTActivity.getUser().getCategoryColor()
\end{source}, which corresponds to the colour that the category list already has for the current user.\newline
If the \textbf{Ok} button is pressed on the AmbilWarna dialog, the colour of the category list will be set to the colour that the user picked in the dialog.

\subsection{Result}
PARROT gives the user the ability to change the colour of different items.\newline

\subsection{Notes:}
Currently, the AmbilWarna dialog is used in the Options fragment %tab/fragment?
to change the colour of some of the items in the Speechboard fragment. It is also used in the Management fragment to change the colour of the categories asociated with a user.\newline
Note that changing the colour of items is only a proof of concept. In the final version of PARROT, the guardians will have the ability to tailor the user interface for the needs of each individual child.\newline

\subsection{Further Reading:}
