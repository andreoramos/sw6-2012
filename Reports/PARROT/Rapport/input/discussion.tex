\chapter{Discussion}
 
This semester we have been working on creating an android application for children with ASD. 
This has brought many new and interesting aspects we have not experienced in former projects. 
The idea of working in a multi project consisting of many groups that need to work together is completely new to us. 
Creating a project we know will be used in real life at some point is new as well, as is the agile development method we have been using. Finally we have not been working with android applications before, which have been rather interesting.\\ 
\\
\subsection*{The multi project}
This semester has differed from other semesters drastically by having a multi project instead of minor individual project. 
This have had a large impact on how the groups have had to work, and made it necessary to cope with problems that we have not encountered before.\newline
The groups are largely dependent on each other as all parts of the product needs to work together. 
We realized this early on, and have taken steps to increase communication between the individual groups. 
One of these steps were to hold weekly meetings, where we presented what we had been working on, as well as what we where going to work on.\newline 
This helped us to constantly know where everyone was at any given time, and helped communication between the groups greatly. 
At this meeting we also shared knowledge between groups, so that problems other groups could have run in to could be avoided by the remaining groups.\newline
Another step that has been taken is to eat together in the cafeteria. This have greatly improved communication between groups. 
Here we could talk about our free time, but we also talked about the projects more common aspects and helped communication on a daily basis without it being too formal.\newline
The dependency of other groups' work have also had a great impact on our work process. 
The good thing is that it has become far easier to specialize on one field. 
For instance have we used a database, but needed to create and maintain it, only ask the group handling the database for functionality. 
One of the problems, though, are that this also means that decisions can be made in the multi project or other groups instead of the individual groups.\newline
The individual groups need to accustom themselves to the fact that they might need to adjust their ideas to that of the multi project, or another group. 
For instance have we in our group wanted some extra tables in the database, but was decided against this.\newline
We believe this way of working give a better view on how work are done in the real world. 
The way we have been working this semester feels like working in a bigger corporation rather than in small individual project groups.  
The fact that the group rooms are shared by two groups only separated by cardboard stands resemble cubicle workstations, which have had decrease in efficiency of brainstorms and discussions, but increased communication across groups.\newline
\\

\subsection*{Agile method - SCRUM:}
Another new experience in this project is that we choose to work with a branch of development methods that we have not work with before. 
This new branch of development methods is the Agile paradigm, and the method that we have chosen is SCRUM. 
Agile methods promote adaptivity, and the idea of changing the plans on the way, working in iterative steps in close contact with the customer, analyzing, implementing and testing along the way. 
It also incorporates the idea of creating one functionality at a time, and adding it to the complete system as it is developed. 
This means that it has been possible for us to make a list of functionalities, and then focus on implementing them one at a time.\newline
We have not implemented them all, but we never expected to be able to do that in the given time frame.
Instead we have made a complete list of functionality that, if implemented, could complete the PARROT application so that it would be ready for deployment.
We hope that future project groups that continue the PARROT project will take this list to heart, and complete the PARROT application.\newline

The SCRUM method involves programming in short-term session known as sprints. 
In these sprints we program the features, which are planned immediately before the sprint during the sprint planning meeting.\newline 
How much effort and time we use for any feature is determined by utilizing planning poker, which means that every member of the PARROT group is given a deck of cards. 
When a feature is listed, each member chooses a card, representing the member's estimate of the amount of work needed for this feature.  
We do this because it helps us organizing a sprint and keeping realistic goals, all within a short amount of time.\newline 
In our work process, this has helped keeping a steady stream of work throughout the whole semester, and giving the group a constant feeling of achieving goals.\newline 
To make sure that we know what functionalities are being worked on, how far we are currently, and generally keep track of our progress, we have assigned a SCRUM master. This person have tried throughout the semester to update a burndown chart of our progress, and copy it out to the multi project bulletin board where all burndown charts are pinned.\newline 
This chart also helps the group to know how far we are, and have often given the group a morale boost.\newline
We have added a few new procedures to the SCRUM method. Instead of a stand up meeting, we have used walk and talk. This has helped us have a fresh mind on subjects, and given us fresh air while talking.\newline
In addition we have added the procedure of Pair programming, which have helped correct mistakes while writing, and have increased performance and quality overall. 
A last procedure we have added is a board, divided into two parts: ``Free'' and ``In Use''.
In these two parts, multiple sticky notes are shown, each representing a class or an xml file. When one is being modified, a note is moved from the Free section to the In Use section under the name of the programmer that is using it. This has helped us avoid revision conflicts.
\\

\subsection*{Real World Assignment}
As opposed to the last semester we have been working on a project that, if luck has it, will end up in the real world.\newline
We have had contact with institutions through contacts that have described the needs of the institution.\newline
This have had a great impact on our work, as we know that one day the project might be used by an institution which helps children with ASD. 
We have had some interviews with these contacts, as well as having been on a visit to the institution to get a better insight of in what context PARROT will be used. 
By this, we have learned the importance of handling pictograms for the children, and we learned that we needed to make sure that it could be used on a simple level, rather than with a lot of fancy features. 



\subsection*{Android}
Working with Android development has been an interesting challenge. First of all, Android incorporates the way of creating a user interface in an easy way, primarily because of the way the graphical components are created in XML.\newline
In comparison with the other projects that the group has taken part of, this project focuses more on creating a graphical user interface, that is functional and safe to use.\newline
To make the design simpler and more organized, we split the design into tabs, each refering to an individual fragment of the functionality. We felt that this was smart, since we could restrict access to the individual tabs, thus ensuring that the children would only have access to the functionality they would need.\newline
Since we were using Android, we would be working with a touch screen input, which required us to rethink our usual design approach, since we were used to having a mouse and keyboard as input.\newline
Finally, working with new technology was interesting, and gave us insight into Android development which we can use for personal projects.