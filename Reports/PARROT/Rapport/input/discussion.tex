Discussion
 
This semester we have been working on creating an android app for children with ASD. 
This has brought many new and interesting aspects we have not experienced in former projects. 
The idea of working in a multiproject consisting of many groups that need to work together is completely new to us. 
Creating a project we know will be used in real life at some point is new as well, as is the agile development method we have been using.
And finaly we hvae not been working with android applications before, which have been rather interesting. 

The multiproject
This semester has differed from other semesters drastically by having a multiproject instead of minor individual project. 
This have had a large inpact on how the groups have had to work, and made it necesarry to cope with problems that where not there before.
The groups have a large dependensies on each other as all parts of the product need to work together. 
We realised this ealy on and have taken steps to increase communication between the individual groups. 
One of these steps where to hold weekly meetings, where we presented what we had been working on as well as what we where going to work on. 
This helped us to constantly know where everyone was at any given time, and helped communication between the groups greatly. 
At this meeting we also shared knowledge between groups, so that problems other groups could have run in to could be avoided.
Another step that have been taken is to eat together in the cafeteria. This have greatly improved communication between groups. 
Here we could talk about our free time, but we also talked about the projects more common aspekts and helped communication an a daily basis wihtout it being to formal.
The dependency of other groups work have also had a great impact on our work process. 
The good thing is that hase become far easier to specialise on one field. 
For instance have we used a database, but needed to create and maintain it, only ask the group handling the database for functionality. 
One of the problems, though, are that this also means that decissions can be made in the multiproject or other groups instead of the indevidiual groups.
The individual groups need to acustom their self to the fact that they may need to adjust their ideas to that of the multiproject, or another group. 
For instance have we in our group wanted some extra tables in the database, but was decided against this.
We believe this way of working give a better view on how work are done in the real world. 
The way we have been working this semester fields like working in a bigger coorporation rather than in small individual projectgroups.  
The fact that grouprooms are shared by two groups only seperated by cardboard stands resemble cubicle workstations, which have had decrease in effeciency of brainstorms and descussions, but increased communication across groups. 

 
 Multiproject
 - m�der
 - vidensdeling
- project dependensies
- better social aspects 
	-lunch
- cubicle work environment
- part of developing a greater product.
- great degree af realism


Agile method - SCRUM
- sprint 
- Scrum master
- Planning poker



Android


RealWorld assignment
- kunde
- interview
- bes�g realworld location, ustomer-developer interaction
- har haft inflydelse p� angegemang



Walk and talk
Free/inUse
Pair Programming
