\chapter{Application Design}
\label{appdes}
Based on the original design for the Digi-Pecs application, modified by the information gathered in the previous chapters, we have come up with the general idea for how we want the design for PARROT to be.\newline
First of all, we want the application to be safe, simple to use, and easy to remember.\newline
By \textit{safe}, we mean that the users of the system should only have access to their own area. For instance, a child user should not (commonly) have access to the options and administration parts of the application, while guardians should not have access to the profiles of other guardians, or children not under their supervision.\newline
For the system to be \textit{simple to use} and \textit{easy to remember}, we want the application to have a relatively simple user interface. Especially because the application is meant to be used by children with autism. Furthermore, the reason for making this application is to make the daily use of pictograms easier. This goes for the guardians as well. If they can't use the system, the children will not be able to learn how to use it.\newline
It is not our thought that the system has to be intuitive, but it is our opinion that if the user is provided by a short demonstration or manual, they will master the concepts of the application in no time.\newline
\\
Designing the user interface has been an iterative process, such is the method of SCRUM. We started by having a look at the design from the report of the previous year (we were unable to see the actual design) and compared it to the tools shown to us by Tove S\o{}by.%skal der refereres ved navn?
We decided that we wanted our design to look and feel like the real thing, so we drew up a quick sketch of how the system would look in our opinion.\newline
\begin{figure}[htbp]
	\centering
		\includegraphics{}
	\caption{Sketch of the speech board tab}
	\label{sketchspeech}
\end{figure}
%indsaet billede
As seen in the picture, we wanted a speech board to show the current available pictograms, a sentence board to put the pictograms on to create a sentences, and a number of tabs representing different categories of pictograms. We also put a play-button on the design, so that the user could press it to play out the sentence.\newline
The reason for this choice was based on the idea of having the system be as close to the existing tools as possible. The guardians already have speechboards and sentence boards, and they have lots of folders containing pictograms organized into different categories. This is why we went for the tab style of the categories, as that would make it look more like a folder.\newline
As we began implementing the system, we realized that we had to make several changes. First of all, the size of the initial sentenceboard was too small, it would have to be bigger in order to make the pictograms easily visible. Also, we had to change the design from tabs to buttons, as we discovered that Android does not natively support vertical tabs. We also removed the play-button, mainly because our contact said that she would rather have the ability to play each pictogram one at a time, than only the ability to play the entire sentence. Also, the button took up a lot of the already limited space on the interface.\newline
\\
After the designing of the speech board, we turn continued the parts of the application that is restricted to the guardians for the children. we had to make the design so that the options and the administrations where easy to access for the guardian when they need to make changes for the specific child that is using the application. 
\\
We tried to get see what the application from last year did but this was not possible so we began to search for other ways to do so, we can up with that we would use tabs so that we would get a tab for the speech board, a tab for options for the colors and a tab for the administrations for categories.\\

The design for the administration for categories we look at have choose to split the tab in three part, first we needed at list over existing categories and a list view to select between the different profiles next a view for all the pictograms that is available the last part is where the information for the category is being given this would which pictogram that should be the button logo, the color for the category and name.  

Now we had been through the design for the application, we also have a lot of features for the application: 

\begin{itemize}
	\item \textbf{Drag and Drop:} This feature allow the user to choose a pictogram and drag it to it's destination and drop the pictogram.  We choose this feature for the application because that we what the application to simulate how the pictograms are used normally, meaning that the user has some pictograms and can "take" a pictogram and put it on the sentence board.  
	\item \textbf{Sound:} This feature enable us to play the audio track that is a part of the pictogram. The feature was chosen because we need a way so the user of the application can play the sound of the pictogram. 
	\item \textbf{Action Bar:} This feature allow us to use tab in the application. We are using this feature so the options and administrator for categories would be easy to access for the guardian. 
	\item \textbf{Color Picker:} This feature gives the user a color scheme so the user can choose which color the speachboard tab should have and the color picker is also used when the guardian chooses color to the categories. we use feature because one of main point in our application is that we can adjust the application to each child. 
	\item \textbf{Showing Pictogram:} This feature takes the pitograms that is a part of the the users profile and put it into the pictogram grid in the speech board tab. this feature is important for it is the part that gives us our pictogram. 
	\item \textbf{Sentence Board:} this part of the speech board is where the user can drag pictogram down onto and arrange the chosen pictograms so the user can make a sentence and it is also where the user can play the audio track in the pictogram. This feature is essential for our application since that is it all this application is about. 
	\item \textbf{Database:} This is where the the profiles and pictogram path are stored on the tablet to be used by the application.
\end{itemize}

So this design see figure\ref{sketchspeech} we got is controlled by our contact Tove and she gave a positive feedback so we put our design in to production.   