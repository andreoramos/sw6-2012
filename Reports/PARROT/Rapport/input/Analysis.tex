\chapter{Analysis}

\textit{In this chapter, we will present the results of the initial analysis phase of our project, the daily use of pictograms, as well as the reason for using our product.}\\
\\
To gain greater knowledge on how to work with children, and which tools are in use in the daily work, we have been in contact with  Tove S\o{}by, a speech therapy consultant from ``Taleinstitutet Aalborg'', who works with children with ASD.\\ 

Based on the meeting with Tove (see notes in \autoref{appendice:notater_fra_tove}), we have learned that one of the tools used in the communication with the children is pictograms.
The pictograms are used in a schema for the day, where all their daily activities are listed as pictograms.
This gives the children the possibility to go to their schema and see what they are going to do next. 
The pictograms are also used for direct communication with the children. The pictograms are used by the children to construct sentences to communicate with others, and also to try to teach them to speak if that is a problem for a specific child.\\
\\
In the use of pictograms, the problem is that they are very space-consuming, and the tools are not very practical to move around, since they consist of a bookcase full of folders with pictograms. 
Another problem is that when new pictograms are needed, the personnel that needs the new pictogram has to print them out, cut them out in small squares, and laminate them. 
So in addition to the folders taking up a lot of space, it is also time consuming to produce new pictograms for the children.\\ 
\\
We have also learned that there are various degrees of autism, and how it affects the individual child. 
This is not in focus of our project, but as the children can have various degrees of autism this means that no two children are the same, as mention in the introduction see \autoref{sec:target}. 
This means that the tools will need to be able to adapt to the needs of each individual child.\\
\\ 
Therefore, based on what we have learned from Tove and from our own research, it would be practical to have a digital version of the pictograms. So that rather than having a lot of boards and a lot of books filled with pictograms for the children, they would only need a few tablets with the same functionality, just converted to a digital version.\\
\\
\textit{We have now presented the analysis, demonstrating the daily use of pictograms, as well as some of the backsides of the tools currently in use. }