\chapter*{Summary}

The main focus of the project is $Android^{TM}$ application development for children with Autism Spectrum Disorders (ASD).  PARROT is a part of the ongoing multi project GIRAF which is currently in its second iteration. The multi project advanced using the agile paradigm, specifically the scrum development method.\newline
\\
The PARROT project application is for children with ASD, designed to help with communication in their daily lives by using digital pictograms. The report describes the analysis and design choices behind the development of the application, as well as a series of tests performed to secure the stability and usability of PARROT.  It also shows how the user can interact with pictograms. The pictograms can be formed into sentences by dragging them onto a sentence board, as a digital version of the tools already in use.\newline
\\ 
The application is a digital version of the pictogram tools, and as such the way the pictograms are displayed can be individualized to suit the needs of each child, by using categories and color. The categories are a way of organizing the pictograms in a way that makes sense for each child.  Additionally, the application color scheme can be adapted to fit the preferences of the children. \newline
\\
PARROT is not yet a finished product, but a basis for further development in the continuation of the GIRAF multi project.

