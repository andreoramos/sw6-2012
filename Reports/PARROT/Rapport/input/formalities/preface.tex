\chapter*{Preface}
This report is written in the sixth semester of the software engineering education at Department of Computer Science, Aalborg University in the Spring of 2012.  \\
\\
The report is written in \LaTeX. and it is documentation for the project, made in the period from first of February until fifth of June 2011.
The semester topic is ``Application Development''. The main goal is to gain knowledge about how to make a application. This is done through designing and developing an application for $Android^{TM}$ based tablets.\\
\\
The reader is expected to have some basic knowledge about application programming and knowledge of how to code in Android. \\
The following are general unless anything else is mentioned.

\begin{itemize}
\item Cites and references to sources are denoted by square brackets containing a number,  like this; \cite{XP}. Sometimes it is followed by a comment used to specify the reference, like this; \cite[Comment]{XP}. The letter and number is corresponding to an entry in the bibliography. 
\item Abbreviations will be presented in extended form the first time they appear. 
\end{itemize}

Throughout the report, the following typographical conventions will be used:
 
\begin{itemize}
\item Omitted unrelated code is shown as `. . . ’ in the code examples.
\end{itemize}

All code examples given in the report are not expected to compile out of context.\\
\\
Appendices are located at the end of the report.\newline
A installation guide for how to install the PARROT application can be found in the \autoref{setup}.
As this is a multi project report, the Introduction chapter is not written by the PARROT group, but written by all of the participants of GIRAF. As such, the opinions expressed in the Introduction may not match those of PARROT.\\ \\ 

This group would like to thank Tove S\o{}by in aiding us in understanding the daily use of the pictogram tool as well as providing an insight into the differences of the children.\\
Additionally we would like to thank Yuku Sugianto for creating the AmbilWarna library and making it available through to the Apache License 2.0. \cite{ambilw} 

  

\begin{tabbing}


The source code for this report is attached on the CD-Rom at \= the last page of the report. A PDF \\
version of the report is also included on the CD-Rom.

\\ \\ \\ \\ 

\textbf{Participants of the project:}

\\ \\ \\ \\ \\ \\ \\ \\ \\ \\ \\ \\ \\ \\ \\ \\
\_\_\_\_\_\_\_\_\_\_\_\_\_\_\_\_\_\_\_\_\_\_\_\_\_\_ \> \_\_\_\_\_\_\_\_\_\_\_\_\_\_\_\_\_\_\_\_\_\_\_\_\_\_
\\
Kim Arnold Thomsen \> Jakob J\o{}rgensen\\ \\ \\
\\
\\
\_\_\_\_\_\_\_\_\_\_\_\_\_\_\_\_\_\_\_\_\_\_\_\_\_\_  \> \_\_\_\_\_\_\_\_\_\_\_\_\_\_\_\_\_\_\_\_\_\_\_\_\_\_
\\
Christoffer Ils\o{} Vinther \> Rasmus D.C. Dalhoff-Jensen\\ \\ \\
\\

\end{tabbing}
