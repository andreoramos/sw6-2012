\chapter*{Preface}
This report is written in the sixth semester of the software engineering education at Department of Computer Science, Aalborg University in the Spring of 2012.  \\
\\
The report is written in \LaTeX. and it is documentation for the project, made in the period from first of February until fifth of June 2011.
The semester topic is ``Application Development''. The main goal is to gain knowledge about how to make a application. This is done through designing and developing a application in android to android based tablets.\\
\\
The reader is expected to have some basic knowledge about programming application and knowledge of how to code in android. \\
The following are general unless anything else is mentioned.

\begin{itemize}
\item Cites and references to sources are denoted by square brackets containing alphabetic letters followed by a number,  like this; \cite{XP}. Sometimes it is followed by a comment used to precise the reference, like this; \cite[Comment]{XP}. The letter and number is corresponding to an entry in the bibliography. 
\item Abbreviations will be presented in extended form the first time they appear. 
\end{itemize}

Throughout the report, the following typographical conventions will be used:
 
\begin{itemize}
\item Omitted unrelated code is shown as `. . . ’ in the code examples.
\end{itemize}

All code examples given in the report are not expected to compile out of context.\\
\\
Appendices are located at the end of the report. \\ \\  \\ \\

This group would like to thank Tove S\o{}by in aiding us in understanding the daily use of the pictogram tool as well as providing an insight into the differences of the children.  

\begin{comment}

This report is written in the fifth semester of the software engineering education at Department of Computer Science, Aalborg University in the fall of 2011.  \\
\\
The report is written in \LaTeX. and it is documentation for the project, made in the period from first of September until twentieth of December 2011.
The semester topic is ``Embedded Systems''. The main goal is to gain knowledge about Real-Time Systems (RTS). This is done through designing and developing a system made of LEGO®
MINDSTORMS® NXT.\\
\\
The reader is expected to have some basic knowledge about programming for real time systems and the design there of. It is not necessary for the reader to have any experience in developing for the NXT, an insight of the concept could be a advantage for the understanding of the report.\\

The following are general unless anything else is mentioned.

\begin{itemize}
\item Cites and references to sources are denoted by square brackets containing alphabetic letters followed by a number,  like this; \cite{ni_lego}. Sometimes it is followed by a comment used to precise the reference, like this; \cite[Comment]{ni_lego}. The letter and number is corresponding to an entry in the bibliography. 
\item Abbreviations will be presented in extended form the first time they appear. 
\end{itemize}

Throughout the report, the following typographical conventions will be used:
 
\begin{itemize}
\item Omitted unrelated code is shown as `. . . ’ in the code examples.
\end{itemize}

All code examples given in the report are not expected to compile out of context.\\
\\
Appendices are located at the end of the report. \\ \\  \\ \\
\end{comment}
\begin{tabbing}


The source code for this report is attached on the CD-Rom at \= the last page of the report. A PDF \\
version of the report is also included at the CD-Rom.

\\ \\ \\ \\ 

\textbf{Partisipators of the project:}

\\ \\ \\ \\ \\ \\ \\ \\ \\ \\ \\ \\ \\ \\ \\ \\
\_\_\_\_\_\_\_\_\_\_\_\_\_\_\_\_\_\_\_\_\_\_\_\_\_\_ \> \_\_\_\_\_\_\_\_\_\_\_\_\_\_\_\_\_\_\_\_\_\_\_\_\_\_
\\
Kim Arnold Thomsen \> Jakob J\o{}rgensen\\ \\ \\
\\
\\
\_\_\_\_\_\_\_\_\_\_\_\_\_\_\_\_\_\_\_\_\_\_\_\_\_\_  \> \_\_\_\_\_\_\_\_\_\_\_\_\_\_\_\_\_\_\_\_\_\_\_\_\_\_
\\
Christoffer Ils\o{} Vinther \> Rasmus D.C. Dalhoff-Jensen\\ \\ \\
\\

\end{tabbing}
