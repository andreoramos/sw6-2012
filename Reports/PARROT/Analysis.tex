\chapter{Analysis}

To get a greater knowledge of how to work with children with autism and which tools are in use in the daily work with the children, we has been in contact with  Tove S�by, a speech therapy consultant that works with children with autism.\\ 

Based on the meeting with Tove, we have learned that one of the tools used in the communication with children with autism is pictograms.
The pictograms are used in a schema for the day, where all their daily activities are listed as pictures. This gives the children the possibility to go to their schema and see what thay are going to do next. The pictograms are also used for direct cummunication with the children. The pictograms are used by the children to contruct sentences to communicate with others, and also to try to teach the to speak if that is a problem for the specific child.\\

In the use of pictogram, the problem is that they are very space-occupying, and the tools are not very practical to move around, since they consists of a bookcase full of folders with pictograms. Another problem is that when new pictograms are needed, the personel that needs the new pictogram has they  print them out, cut them out in small squares, and laminate them. So in adition to the folders taking up a lot of space, it is also time consuming to produce new pictograms for the children. 

We have also learned that there is various degrees of autism, and how it affects the individual child. This is not in focus of our project, but as the children can have various degrees of autism this means that the no children are the same, as mention in the introduction,!s�t ref til 1.2 target group!. This means that the tools will need to be able to adapt to the needs of each individual child.
 
Therefore, based on what we have learned from Tove and from our ovn research, it would be practical to have a digital version of the pictograms. So that rather than having a lot of boards and a lot of books filled with pictograms for the children, they would only need a few tablets with the samne functionality, just converted to a digital version.
