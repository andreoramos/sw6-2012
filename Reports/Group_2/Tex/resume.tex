\chapter*{Summary}
The GIRAF multi project system is a continued development of a multi project, started by the bachelor students of 2011. The GIRAF system is specifically developed to ease the daily routine for educators and parents of children with autism spectrum disorder by providing a collection of tools, which consist of an administration module (Oasis), a server with a web interface (Savannah), a launcher (GIRAF Launcher), a speech board (PARROT), and a timer application (WOMBAT). GIRAF is built for the Android\texttrademark \hspace{0.1cm} operating system, specifically for the Samsung Galaxy Tab 10.1 and the Android 3.2 platform.\\

   The project \textit{WOMBAT: Part of the GIRAF System} focus on developing an application, which can visualize progressing time, by mimicking the physical timers already used by parents and educators of children with autism spectrum disorder. The WOMBAT application is designed in cooperation with an educator from Birken, a kindergarten for children with autism spectrum disorder, in Langholt.\\
	
	The study regulation demands that the project is developed as a multi project. Therefore the projects had to be flexible during the development period and the groups agreed to use the development method Scrum of Scrums.\\

   In order to validate the usefulness of the application, it was both usability tested and acceptance tested. These tests prove that the application is usable by both educators as well as the children they work with.
\clearpage