\chapter{Development Process}
As stated in chapter \ref{cha:project_management}, the development method used in this project is a modification of SCRUM, which means that the development evolve through sprints. Here is a description of the six development sprints we had, and in appendix \ref{sec:burn_back} all sprint backlogs can be found.

\section{Sprint Walk-through}
Because it took some time for all groups to agree on a development method, we started making some design and talking to our contact person before the actual sprints started, so when the first sprint was started, we already had an idea of the functionalities we wanted to end up with.

\subsection*{Sprint 1: 19/03 - 23/03}
In the first sprint, the main objective was to set up the android project in development environment, and get the development started. At this point we were expecting to end up implementing the whole vision (see section \ref{sec:vision}), and a natural place to start would be the timer. We had an idea of how the design should be, and we began to implement lists for holding children and configurations.

\subsection*{Sprint 2: 26/03 - 04/04}
In the middle of this sprint the launcher group stated that they would only be able to make the guardian mode in the launcher in this semester, so we decided in the timer group, that the two last parts of the timer application, the overlay and the settings, would not be developed in this semester either, since there will be no need for them if there is no child mode in the launcher.\\
	We continued designing and developing, and in the end of this sprint, we had the design ready for implementation. We also started to make some OpenGL development, as we thought that OpenGL was the best way to implement the timers.\\
	We had a meeting with the contact person, and she suggested that when the time has run out, it would be possible to show two custom pictograms on the screen. Also she suggested that two different timers could run in parallel on the screen, if a child was used to one specific timer, but was learning how to use another type of timer. We added these to suggestions to the product backlog.

\subsection*{Sprint 3: 10/04 - 19/04}
We found out during this sprint, that it was more convenient to implement the timers using canvas and 2D drawings. The progress bar had been implemented in both OpenGL and on canvas, and the canvas version was the easiest to implement, and it was on fewer lines of code.	

\subsection*{Sprint 4: 23/04 - 04/05}
In this sprint we finished drawing the timers, and a lot of the most vital functionalities, such as loading and saving configurations, highlight on list items, and the "Done" screen, were implemented.\\
	During this sprint we had a lot of contact with the admin group, because we had difficulties implementing functions to load and save data to the database. We also integrated the timer with the launcher, and did some testing on the interaction between them.

\subsection*{Sprint 5: 07/05 - 11/05} 
This sprint was used to do some refactoring of the code, to make it more readable and understandable. We also implemented the last things from the backlog, and did some polishing to the overall design. Furthermore we started making test design and test cases for the functionalities we want to test in the timer application.

%	\subsection*{Sprint 6: 14/05 - 18/05}
%	This sprint...