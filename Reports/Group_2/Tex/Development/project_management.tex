\chapter{Development Tools}
	\label{cha:development_tools}
Beskrivelse af miniprojektet fra SOE kurset (om udviklings metoder etc)
Hvilke værktøjer har vi brugt fra SOE kurset
- Beskrivelse af disse!

Fra soe...

%\chapter{Project management}
%This project is a multi-group project, and five groups of three to four people are working together to develop one combined product. For this to work out, it is necessary to have a development method which suits this kind of multi-group development. Through the course \textit{Software Engineering}, a part of our study regulation, we learned about SCRUM, and we chose to adopt this technique.\\
%Section \ref{sec:mod_of_scrum} is based on a mini-project written in the course \textit{Software Engineering}, and it explains the modifications we have applied to the SCRUM method.
%
%\section{Modification of SCRUM}
	%\label{sec:mod_of_scrum}
%Even if we decided that SCRUM would best suit our situation, we had to modify the method to make it fit better. The modifications are as follows:
	%\begin{itemize}
		%\item Daily SCRUM is substituted with weekly SCRUM.
		%\item We have no SCRUM master. The group is together responsible for the SCRUM master's tasks.
		%\item Extensive documentation is done, since the bachelor project next year is going to be based on the results of this project.
		%\item Instead of a project owner we have a democracy between all groups. This is because everyone has to learn from this project and therefore everyone has to be engaged all phases of the development.
		%\item We are limited in time, so we are probably only going to do three 2-week sprints in total.
	%\end{itemize}
%
%\paragraph{Arguments for Chosen Method}
%Since this project requires the entire android group to be split into smaller groups of 3-4 people, we are going to use a development method which supports small groups working together.
   %It is important that we develop prototypes, showing the project concepts in progress, to make sure we reach a usable final product.
   %Since everyone involved in the project is novices in the field, students being new to working with autistic children and educators being new to working with software developing, an agile method would be suitable, as changes in demands to the product might occur often.\\
%
%Agile development meets that the software is developed as part of a learning process.
%
%\section{Versioning}
%For versioning the code and documentation, we use a SVN-server at \textit{Google Code}\cite{web:googlecode}.\\
%
%There several folders on the server, which help the groups to keep track of working code, reports, etc.. The structure is as follows:
	%\begin{itemize}
		%\item \textit{trunk} - contains iteration releases of the code.
		%\item \textit{branches} - contains code under development.
		%\item \textit{tags} - contains the newest final release of the code.
		%\item \textit{common\_report} - contains the common part of the report.
		%\item \textit{Reports} - contains all group reports.
		%\item \textit{wiki} - contains summaries from all group- and supervisor meetings, and guides/agreements.
	%\end{itemize}