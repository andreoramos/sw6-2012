\clearpage
\section{Test Results}
	
	\begin{table}[width=\textwidth]
		\begin{center}
			\begin{tabular}{|l|c|c|p{6cm}|}
				\hline
				\textbf{Test Case ID} & \textbf{Pass} & \textbf{Fail} & \textbf{Notes} \\
				\hline
				\textit{saveAs\#1} & \checkmark &  &  \\
				\hline
				\textit{save\#1} & \checkmark &  &  \\
				\hline
				\textit{checkLastUsed\#1} &  & \checkmark & When tablet memory is cleared, the last used timers disappears. \\
				\hline
				\textit{hOnClick\#1} & \checkmark &  &  \\
				\hline
				\textit{stillHAfterSave\#1} & \checkmark &  &  \\
				\hline
				\textit{hChildOnLaunch\#1} & \checkmark &  &  \\
				\hline
				\textit{checkTimerTime\#1} & \checkmark &  &  \\
				\hline
				\textit{checkDoneFunc\#1} & \checkmark &  &  \\
				\hline
			\end{tabular}
			\caption{Test results of the black box testing of the WOMBAT application.}
			\label{tab:bb_test_results}
		\end{center}
	\end{table}

\subsection{Reflections}
We did not outsource the tests, therefore did we conduct the tests ourselves, this is not the proper way to perform black box testing. 
This means that the testers also developed the application, the test design, and the test cases, and thereby have knowledge of how the features are implemented.
Therefore may a test person without knowledge of the code have another approach to the tests than what we have.

We already knew that \textit{checkLastUsed\#1} would not pass before conducting the tests. 
We had trouble implementing the "Last Used" feature with the OasisLocalDatabase, this resulted in sporadic system failure.
We decided to use the tablet memory to save the last used configurations.