\chapter{Development Tools}
	\label{cha:development_tools}
	As part of a mini project in the course \textit{Software Engineering} we analyzed the project situation, when we had started the development process, by listing strengths, weaknesses, threats, and opportunities of the project, and discussed if some of the threats and weaknesses could be eliminated, or some strengths or opportunities could be exploited at a low cost. We here list the development tools we decided to use, and the reason for choosing them, the analysis of the project can be found in \autoref{sec:swot}.
	
	\section*{Pair Programming}
		We implemented this technique because we had done some single person programming in the start of the development process, and there was a growing need for explanation when using methods the other project members had written. Also we wanted to enhance the quality of the code we wrote, and thereby minimize the time we had to spend on debugging.

The cost of this technique was, in this case, moderate, since the project group is very small, and it makes a difference when one person is not writing code. On the other hand, the code written in pairs is not likely to be re-written, because of the higher quality, which makes the benefit high.
	
	\section*{Refactoring}
	The reason refactoring was a usable technique for our project, is that we were suffering from code cluttering and code generally being difficult to understand. When debugging the system it could sometimes be difficult to know where the bug had evolved from. Since refactoring can give a better overview it would make debugging an easier task. Also this project is going to be continued by another project group next year, which requires our code to be easy to understand for the next project group to be able to smoothly proceed the development.

The cost of this technique is low, and the benefit is, for ourselves, moderate. The benefit is much higher for the project groups next year.