\section{Acceptance Test}
\label{sec:accept_test}
The purpose of the acceptance test\cite{misc:designInterSys} is to test if the system fits into the context it is designed for. 
The main objectives is to know if the interface feels natural for the guardians to use the timer application instead of their regular physical timers, and to know if the children understand the digital version of the different timers.\\

To test for acceptability, the application needs to be tested in the context it is developed for, therefore did an educator, at the kinder garden Birken, borrow the tablet with the timer application for a few days, such that she could use the system in real life scenarios. 

\subsection{Results and Observations}
The test person wrote a diary, to keep track of her experiences with the application. The test person used the timer application herself, and she let some of the other educators try it as well. The diary can be found in appendix \ref{sec:acceptance}.

The diary explains, that the children understood the meaning of both the timers and the pictograms. One of the children was more interested in watching the digital timer countdown than in the activity he was meant to do, but beyond that there were no problems with the application.\\

We can tell by the last used list, found in \autoref{sec:acceptance}, that the test person used different timers with different pictograms and timespans, only one out of eight timers had a slightly different color. This could indicate that changing the colors on the timers is not very intuitive, or they do not need the functionality.