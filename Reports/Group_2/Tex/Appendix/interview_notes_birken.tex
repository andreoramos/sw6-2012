\section{Notes from Interview}
\label{InterviewMette}
\textit{This is notes from an interview with Mette Als Andreasen, an educator at Birken in Langholt, Denmark.}

Naar tiden loeber ud (kristian har tage et billede):\\
Faerdig - symbol\\
Gaa til skema - symbol\\
Taget fra boardmaker\\

Kunne vaere godt hvis man kunne saette egne billeder ind som start/stop symboler.\\


Roed farve $=$ nej, stop, aflyst.\\

De har saadan et ur paa 60 minutter hvor tid tilbage er markeret med roed, og saa bipper den lige kort naar den er faerdig.\\
  Det ville vaere fint hvis de kunne bruge sort/hvid til dem der ikke kan haandtere farver, men ogsaa kan vaelge farver.\\

Stop-ur:\\
en fast timer paa 60 minutter $+$ en customizable som ikke ser helt magen til ud, som f.eks, kan vaere paa 5, 10 eller 15 minutter for en hel cirkel.\\

timeglas:\\
skift farve paa timeglassene, men ikke noedvendigvis goere dem stoerre. Kombinere med mere/mindre sand. Eventuelt kombinere med et lille digitalt ur, til dem der har brug for det, skal kunne slaaes til og fra.\\

Dags-plan:\\
ikke saerlig relevant til de helt smaa og ikke saerligt velfungerende boern. Men kunne vaere rigtig godt til de lidt aeldre.\\
   En plan gaar oppefra og ned, og hvis der saa skal specificeres noget ud til aktiviteterne, saa er det fra venstre mod hoejre ud fra det nedadgaaende skema.\\

Til parrot:\\
Godt med rigtige billeder af tingene, som paedagogerne selv kan tage, eventuelt ogsaa af aktiviteter, saa pedagogerne kan have billeder af aktiviter som de kan liste efter skeamet.\\

Der var mange skemaer rundt omkring, og der henviser det sidste billede i raekken til naeste skema, som haenger f.eks. paa badevaerelset eller i garderoben.