\section{Notes from Interview}
\label{InterviewMette}
\textit{This is notes from an interview with Mette Als Andreasen, an educator at Birken in Langholt, Denmark.}

N�r tiden l�ber ud (kristian har tage et billede):\\
F�rdig - symbol\\
G� til skema - symbol\\
Taget fra boardmaker\\

Kunne v�re godt hvis man kunne s�tte egne billeder ind som start/stop symboler.\\


R�d farve $=$ nej, stop, aflyst.\\

De har s�dan et ur p� 60 minutter hvor tid tilbage er markeret med r�d, og s� bipper den lige kort n�r den er f�rdig.\\
  Det ville v�re fint hvis de kunne bruge sort/hvid til dem der ikke kan h�ndtere farver, men ogs� kan v�lge farver.\\

Stop-ur:\\
en fast timer p� 60 minutter $+$ en customizable som ikke ser helt magen til ud, som f.eks, kan v�re p� 5, 10 eller 15 minutter for en hel cirkel.\\

timeglas:\\
skift farve p� timeglassene, men ikke n�dvendigvis g�re dem st�rre. Kombinere med mere/mindre sand. Eventuelt kombinere med et lille digitalt ur, til dem der har brug for det, skal kunne sl�es til og fra.\\

Dags-plan:\\
ikke s�rlig relevant til de helt sm� og ikke s�rligt velfungerende b�rn. Men kunne v�re rigtig godt til de lidt �ldre.\\
   En plan g�r oppefra og ned, og hvis der s� skal specificeres noget ud til aktiviteterne, s� er det fra venstre mod h�jre ud fra det nedadg�ende skema.\\

Til parrot:\\
Godt med rigtige billeder af tingene, som p�dagogerne selv kan tage, eventuelt ogs� af aktiviteter, s� pedagogerne kan have billeder af aktiviter som de kan liste efter skeamet.\\

Der var mange skemaer rundt omkring, og der henviser det sidste billede i r�kken til n�ste skema, som h�nger f.eks. p� badev�relset eller i garderoben.