\section{Rapport structure}

This rapport is a little different of what we are used to because the work form of the whole project have been different in two ways. First off there have been two groups the little project group developing the launcher and the big mini-project group which contained all the groups developing to the android platform. Secondly this project have not been following the waterfall method but instead been using agile development hereunder scrum and XP.
Therefore there have been some different solutions to how the rapport should be structured. The most obvious choice is the old waterfall rapport model where there is an introduction, an analysis, some description about the design and implementation phase and lastly a conclusion. The benefit of this method is that everything about a subject is in the same paragraph, is fast to find and read. The disadvantage is that it does not show that it was an iterative project that was made.
The other idea was a more log orientated form where there would be explanation about the significant decisions there happened doing sprint 1, sprint 2 and so forth. This would be better at show that it was an agile development and learning process that happened with the project and in the group. The big plus with this approach is that it would be an iterative rapport showing an iterative process and product. The downside is that it fast can make the rapport feel unstructured.
Another thing is the choice that was made in the group about programming first and writing after. This should affect the choice of what style this rapport should end out with. This style talks most for the traditional way where only the major choices is written about where it can be difficult to remember what happened in sprint 1, sprint 2 and so forth.