\section{Logo Activity}
The logo activity is a splash screen, which shows the \giraf[] logo while the system is the loading the authentication activity. The logo activity is justified by the possibility that the device running the launcher might spend more than a reasonable amount of time to load the authentication activity, and therefore might give the impression that the system had crashed or is not responding. The goal is to increase responsiveness. 

However, the logo activity does not indicate whether or not the activity itself has crashed. This is a feature which is on the backlog. \todo{husk at tilfoeje til backlog + suggestion: solution could be create an animation}

\begin{lstlisting}[style=sourceCode, language=JAVA, caption=LogoActivity.java, label=lst:logoActivity] 
		if (Tools.sessionExpired(mContext)) {
			            		intent = new Intent(mContext, AuthenticationActivity.class);
			            	} else {
			            		intent = new Intent(mContext, HomeActivity.class);

			            		SharedPreferences sharedPreferences = getSharedPreferences(Data.TIMERKEY, 0);
			            		long guardianID = sharedPreferences.getLong(Data.GUARDIANID, -1);
		            			intent.putExtra(Data.GUARDIANID, guardianID);
			            	}
\end{lstlisting}

% This section will explain the activity \textit{Logo Activity}\todo{Remember to explain that activity refers to Android Activity} and how it is utilized.\\\\
% This activity is only seen by the user when the launcher is not already on the Back Stack\todo{Android ref}.\\
% The important thing about this activity is that it runs in a \verb+Thread+ and uses \verb+synchronized()+ to wait an amount of time until the Authentication Activity or the Home Activity is most certainly done. Which activity that should be started is decided by the \verb+Tools.sessionExpired(mContext)+ method. 
% This method returns true or false over if the users login is still valid or not. If the login is still valid and the id which is still logged in, in the \verb+sharedPreferences+ do not return \verb+null+.
% It will only return null if the user do not exsist in the system. The user should be taken to the Home Activity else they should be presented with the Authentication Activity so they can login.
% When the right \verb+Intent+ is build, the activity is started and the thread is stopped ad finish.
% All this can be seen in \autoref{lst:logoActivity} and the final design of the implementation can be seen in \autoref{fig:logo-activity_1} and \autoref{fig:logo-activity_2} on page \autopageref{fig:logo-activity_1}\\
% Unfortunately due to errors in the Oasislib the out commented code was not tested.
% 
% \begin{lstlisting}[style=sourceCode, language=JAVA, caption=The Logo Activity and how it uses Thread and synchronized, label=lst:logoActivity] 
% 		mLogoThread = new Thread() {
% 	        @Override
% 	        public void run() {
% 	            try {
% 	            	synchronized(this) {
% 	            		wait(Data.TIME_TO_DISPLAY_LOGO);
% 	            	}
% 	            } catch(InterruptedException e) {}
% 	            finally {
% 	            	Intent intent;
% 
% 	            	if (Tools.sessionExpired(mContext)) {
% 	            		intent = new Intent(mContext, AuthenticationActivity.class);
% 	            	} else {
% 	            		intent = new Intent(mContext, HomeActivity.class);
% 	            		
% 	            		SharedPreferences sharedPreferences = getSharedPreferences(Data.TIMERKEY, 0);
% 	            		long guardianID = sharedPreferences.getLong(Data.GUARDIANID, -1);
% 	            		
% 	            		/* Following did we not have time to test due to errors in the Oasislib */
% 	            		
% 	            		//if ((new Helper(mContext)).profilesHelper.getProfileById(guardianID) != null) {
% 	            			intent.putExtra(Data.GUARDIANID, guardianID);
% 	            		//} else {
% 	            			//intent = new Intent(mContext, AuthenticationActivity.class);
% 	            		//}
% 	            	}
% 	            	
% 	                startActivity(intent);
% 	                stop();
% 	                finish();
% 	            }
% 	        }
% 	    };
% \end{lstlisting}