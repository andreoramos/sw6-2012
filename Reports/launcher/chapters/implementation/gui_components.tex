\section{GIRAF GUI Components}
\label{implementation:gui_components}

\emph{\guicomponents[]} is the name of the library, which is the implementation of extensions to Android components, such that they satisfy the standards explained in \autoref{design:overview}. \\

\noindent The library consists of extensions of standard Android classes:

\begin{itemize}
	\item android.widget.Button
	\item android.widget.BaseAdapter
	\item android.app.Dialog
	\item android.widget.ListView
	\item android.widget.TextView
	\item android.widget.ImageView
	\item android.os.Handler
\end{itemize}


\begin{figure}[h]
	\centering
	\Tree [.Button [.GButton [.GIconButton [.GVerifyButton ] GCancelButton ] ] ]
	\caption{Inheritance of \giraf[] buttons}
	\label{fig:gbuttoninheritance}
\end{figure}

\autoref{fig:gbuttoninheritance} shows the following classes, which directly or indirectly implement extensions to \verb+android.widget.Button+.

\paragraph{GButton} implements coloring, according to the the design language, presented in \autoref{design:design_language}.

\paragraph{GIconButton} implements functionality which allows for the use of an icon inside the button. An icon could be a status icon, as described in \autoref{design:state_icons}.

\paragraph{GVerifyButton} simply uses functionality from \verb+GIconButton+ to set the icon to the \emph{accept} status icon, from 
\autoref{design:state_icons}.

\paragraph{GCancelButton} sets the icon to the \emph{reject} status icon from \autoref{design:state_icons}, by using functionality from \verb+GIconButton+. \\

\noindent The principle of extending android classes is applied to all the classes listed earlier in the itemize, and the the inheritance for all components in the library are shown in \autoref{appendix:guiinheritance}.
