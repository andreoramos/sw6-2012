\chapter{Discussion}
This section will discuss the decisions made throughout the project, and their relevance and ramifications. \newline

The management of the multi-project was conducted by the \localgroup[]s, and no single project manager was chosen.
Having no leader role in the project meant that excessive time was spent making decisions, as a majority of the \globalgroup[] had to be convinced of the quality of a proposal before it could be accepted.
In this scenario, a leader might have been able to cut down the time spent discussing, and force a decision, when excessive discussion hampers decision making.

A possible drawback of having a leader, is if he or she overrules too many discussions and diminishes the involvement of the \globalgroup[] as whole.\newline

In this semester, the \localgroup[]s have a hard limit member count of four \citep{web:rammestudyreg}, instead of the previous limit of six.
The lowered amount of group members demands more effort from each member.
Absence also affects the group in a greater way, as group size lowers. 
This could be accomodated for by reducing the scope of the project.\newline

The backlog is a crucial element in agile development and it is used to encourage developers and help communication across the multi-project.
Backlogs were created in each \localgroup[], but a global one was not created, for the \globalgroup[].
As seen in \autoref{iterative:sprint1}, there were confusion with what tasks each \localgroup[] were handling.
An effort to develop a global backlog might have helped unite and create a clearer target for the \globalgroup[], and helped promote cooperation.
For example, the Savannah group might have prioritized synchronization between Oasis and Savannah higher.
This might have eased the process of testing, as each group would not have to create their own dummy data for their local Oasis database.\\

A goal in XP is to motivate the team members by letting them choose their own tasks, so they can pick the tasks they feel motivated about.
This can however reduce motivation, e.g. bad mood, when no one in the team feels compelled to do a certain task.
Working in pairs might solve the problem, as our experience with pair programming makes us believe that working in pairs could lessen the burden of completing a tasks which otherwise noone would be compelled to do.\\

Communicating the progress in each \localgroup[] was done by using burn down charts, amongst other things, e.g. meetings.
The burn down charts were hard to get accurate readings from, as each \localgroup[] had their own measurement of work load.
It might not be needed to have burn down charts for further development. \\

The essence of working iteratively is the fact your previous work might change.
To accommodate this we decided to write our report after implementation was done. This reduced overhead of keeping the link between the implementation and the report up-to-date.
However, writing the report after the development process ends for us, creates an issue which also occurs in the  waterfall development model.
The issue is in regards to scheduling, as delays in one part of the schedule reduces the time available to work on remaining parts.

	% - Project manager? !
	% - Group size (Sickness ! 
	% - Global feature prioritizing / Vision vs. Feature !
	% - Isolation? (Savannah) !
	% - Task distribution? / Why pair is important and not solo !
	% - Burndown charts? / Needed? !
	% - Writing after iteration? !
	% 
