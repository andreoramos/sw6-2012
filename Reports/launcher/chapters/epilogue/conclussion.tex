\chapter{Conclusion}
In this project, several groups have shared a problem statement, and worked in a multi-project, where they each built their own part of a system that would solve the issue presented. 
The problem statement for the project was:    
 
\begin{quotation}
How can we ease the daily life for children with ASD and their guardians, while complying with the study regulation? 
\end{quotation}

We have designed and implemented a launcher component for the multi-project's Android solution called \emph{GIRAF}. The launcher component deals with: authenticating users, launching apps and allows the user to read and change platform settings. 

We began by gaining understanding of the customer domain. This gave us insight into how the software quality aspects should be prioritized, and their importance to the customers. 
This lead to envisioning through the creation of prototypes, and sketches, prior to the actual design of the launcher. Usability was found as one of the key aspects of the project during this period.
Based on this analysis work, development began with the first sprints focused on the design of the user interface, to ensure that usability was prioritized in the product. The later sprints were focused on implementing core features, and integrating the previously created design.

We implemented a library containing UI components for the other \localgroup[]s to use, in order to enhance usability by creating consistency. In the implementation, we also focused on creating maintainable and readable code, to ensure that the project could be used by other developers.

To verify the quality of the product, we conducted a usability test. The test subjects consisted of the customers of the multi-project. The test highlighted some issues in the launcher, which could be corrected by the next group of developers.