\section{Initialization}
\label{design:initialization}

\subsection*{Requirements}
It might be cumbersome to authenticate every time the \giraf[] launcher is executed, if the user is switching between multiple launchers. Therefore, the launcher must be accessible in a sufficiently fast manner, such that the launcher is not perceived to be cumbersome.

\subsection*{Solution}
In order to speed up the process of switching between different android launchers -- should more than one be installed -- we choose to allow users to access the launcher with the last logged in user, if authorization happened within the last eight hours.

One could argue that this feature -- also described in \autoref{backlog:autologin} -- introduces a security issue, as an attacker, if able to physically get a locked device, running the GIRAF launcher in an authenticated session, could simply reboot the device in order to get full access.

We deem that this is not critical, since a simple workaround is to always logout whenever the device is placed in an location where unauthorized users might get a hold of it.

In \autoref{fig:state_diagram} the ``Uninitialized''-state is the first state reached, when running the launcher for the first time. There are two transitions from the ``Uninitialized''-state:

\begin{itemize}
	\item Initialize
	\item Autologin
\end{itemize}

The \emph{initialize} is taken, if one of the following conditions are met: 

\begin{itemize}
	\item Authentication has never been done before on the device
	\item Authentication was done more than eight hours ago
\end{itemize}

Taking the \emph{initialize} transition brings the launcher to the ``no credentials provided''-state. \\

The second transition -- \emph{autologin} -- is taken in the case authentication was done less than eight hours ago, and brings the launcher to the ``no app selected'' state, such that the authentication process is omitted for the user, although the last authenticated user is still authenticated.