\section{Initialization}
\label{design:initialization}
The Android operating system can preempt any app at any time \citep{android:activity}, and it might be cumbersome to authenticate often, if the \giraf[] launcher is executed after a preemption. 
To improve the user experience, the launcher must be accessible in a sufficiently fast manner after a preemption.

One could argue that this feature introduces a security issue, as an attacker, if able to physically get a device which has been authenticated within the last eight hours, could start the launcher in order to get full access.

We deem that this is not critical, since a workaround is to always log out whenever the device is placed in an location where unauthorized users might get a hold of it.\\

In \autoref{fig:state_diagram}, \emph{Uninitialized} is the first state reached. 
There are two transitions from \emph{Uninitialized}:

\begin{itemize}
	\item Initialize
	\item Autologin
\end{itemize}

The transition \emph{initialize} is taken, if one of the following conditions are met: 

\begin{itemize}
	\item Authentication has never been done before on the device
	\item Authentication was done more than eight hours ago
\end{itemize}

Taking the \emph{initialize} transition brings the launcher to \emph{No credentials provided}. \\

The transition \emph{autologin} is taken in case authentication was done less than eight hours ago, and brings the launcher to \emph{No app selected}, such that the authentication process is omitted for the user, although the last authenticated user is still authenticated.