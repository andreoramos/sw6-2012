\section{Code Testing}
This section presents the testing that had been planned for the product. \newline

Though having testable code has not been a priority (see \autoref{product_quality} for more information), preparations were made in case there was time for doing dynamic testing. 
These preparations revolved around dynamic white-box testing, and includes some test cases for many different functions in the code that would allow for some unit testing to take place. 
These test cases can be seen in \autoref{test_cases}. 
A few test cases were run as a proof of concept, but these were simple cases, that proved we did not have the time to run more extensive testing. \newline
Had there been time to run these unit tests, the next step would have been to run mutation analysis on them to see if the existing cases exercised the code well enough, or more cases would have to be added. \newline

But while there has not been much time for dynamic testing, static testing has been employed vigorously, though not in a formal fashion. 
Static black-box testing has been employed when the design of the product has been discussed, and the design has gone through many iterations and refinements because of it. 
Discussions about the design have been common, though more so early in the project, and iteration has been important for all parts of the product, as is also supported through the agile working process employed. 
See \autoref{launcher_work_form} for more details on the working process. \newline

Static white-box testing has been used as a part of the pair programming and the refactoring employed in the project. 
Both of these have code reviewing as an essential component. 
Pair programming is about creating higher quality code by having real time reviews of the code as it is written, and refactoring is about reviewing code and improving it when bugs are found, the readability of the code is too low, or the complexity of the code is too high.