\section{Usability for children}
\label{Preanalysis:Usability_for_children}
This section builds upon the book "`The Role of Usability Research in Designing Children's Computer Products"' and will clarify what was done to make usability high in the product, and the reasoning behind it.\\

The \giraf[] system is designed to be usable by both children and adults, making the age range of potential users very large, as it can be used by young children all the way up to elderly guardians. 
This is amplified by the fact that the children using the system have ASD, which can result in their mental capacity being below that associated with their physical age. \newline
One of the things that were done was to be sure that both children who are impatient and children who have a high level of patients can use the product. 
This was done because of ``The product needs to accommodate children who click madly around a screen as well as those who sit back and wait to be told what to do.'' \citep{microsoft:usability} and ``Children are often inpatient and can have difference focusing on one task'' \citep{microsoft:usability}.\\
It was important to make the launcher recognizable and the \giraf[] GUI library helped with this because e.g. all buttons in the \giraf[] system are with rounded corners and have some kind of text or a picture inside them. 
Another thing that was done with the buttons is that they were made larger than normal. 
``The results also provide further evidence that young children require interactions designed specifically for their developing motor skills.'' \citep[p. 8]{mousesize}.
This decision was made upon ``They are also better at recognition over recalling'' \citep{microsoft:usability}. 
The \giraf[] GUI library was made to create consistency i.e. make all interactive elements in the \giraf[] system as a hole more recognizable. 
It was also tried to create higher usability by giving the possibility for every guardian and child to have their own icon colors for the programs in the \giraf[] systems, hopefully these ten predefined colors will help the recognition. 
The launcher also tries to somewhat stimulate the everyday life of both children and guardians e.g. the drawer is made to function as a real drawer where it is possible for the user to hide things when the drawer is closed and see what is inside when it is open. 
The calendar icon looks like a normal calendar on paper and many electronic ones again to accommodate with real life situations. 
``Design icons to be visually meaningful to children. 
The best icons for children are easily recognizable and familiar, representing items in their everyday world''\citep{microsoft:usability}.  \\
The launcher includes a hot fix for the back button in the bottom left corner designed by Samsung on their tablets running with their ROM. 
This button lets the user move back and eventually quit the program which is in context. 
This has been fixed so it is impossible to get out of the launcher this way.
``Use a Return button separately from a Quit button. When both are on the screen simultaneously, children tend to choose the Quit button to exit activities and then accidentally exit the program'' \citep{microsoft:usability}. 
Now the back button only serves the back functionallity and can not quit the launcher.

\todo{Should this section be here?}