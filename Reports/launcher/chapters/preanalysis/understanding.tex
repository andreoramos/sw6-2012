\section{Understanding}
This section will address the tools used and the work done for understanding the customer domain.

\subsection{Interviews}
One of the method used for gathering understanding of the customer domain is interviews. 
The customer domain contains two kind of users: \guardian[c]s and \autists[].
\guardian[c]s have extensive\todo{Hate this word} knowledge about the \autists[], and as they are the customers, we have chosen only to interview them.
The structure of the interviews were semi structed, in order to create a solid base of questions, while having flexibility, incase the need of diving into subtopics\todo{DIEB reference}. 
Two types of interviews was performed: One conducted by non-fixed subgroups of the multi-project with the customers divided between each of the subgroups, and the other type was conducted by each of the \localgroup{}s with each of their assigned customer. 
The first type of interviews was done at the university building, in separate rooms. 
The second type of interviews was performed at the customers workplaces. 

The results from the first interview showed: Usability is critical, based on experience from previous software and the need for certificates in order to use the existing software in the same category. Flexibility is also critical, as the needs of each \autist[] is individual.

\subsection{Field Observations}
As a part of the interviews conducted at the customers workplace, the environment was observed. The observation gave insight in: currently used physical and software tools, the robustness of the environment and the organization needed for the \austists[].

The impressions of the field observations were: The current software tools have limited flexibility and have an extensive\todo{hate this word} cost. The physical tools used are robust and fulfill simple tasks. Some of the simple tasks the physical tools fulfill, could easily be solved by software solutions. The environment have the same robustness as a public school, as the same interior is used. The organization is done by the \guardian[]s with the \autists[] having influence, and tools help the communication and enhance the understanding of the \austits[].

\todo{Mangler TAIL}
