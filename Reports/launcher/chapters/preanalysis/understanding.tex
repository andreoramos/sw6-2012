\section{Understanding}
\label{sec:understanding}
In order to understand the customer domain, interviews were performed.
%This section will address the tools used and the work done for understanding the customer domain.


\subsection{Interviews}
\label{interviews}
%One of the methods used for gathering understanding of the customer domain is interviews. 
The customer domain contains two kind of users: \guardian[c]s and \autists[]. 
We set the scope of this project to include the \guardian[]s as primary users, with the \autists[] being only peripheral users. 
For this reason, the \guardian[]s are the customers of the \giraf[] launcher, and are at the same time expert users of the customer domain. 
To gain good understanding of the customer domain, each of the customers available in the project, were interviewed. 
The interviews were semi structured, in order to create a solid base of questions, while having flexibility, in case exploring additional topics would become relevant \citep[page 152]{dieb-book}. 
Two types of interviews were performed: One conducted by non-fixed subgroups of the multi-project with the customers divided between each of the subgroups, and the other type of interviews were conducted later by each of the \localgroup{}s with their assigned customer. 
The first type of interviews were done at the university, in separate rooms. 
The second type of interviews were performed at the customer's workplaces. \newline

The results from the first interview round showed that usability and flexibility are critical for the customers. 
Usability was requested based on experience with previously used solutions, including expensive learning courses, and flexibility was deemed critical, due to individual \autists[] potentially having very different needs.

\subsection{Field Observations}
As part of the interviews were conducted at \egebakken{}\footnote{An institution for \autists[] and the workplace of our customer.}, observations were also made while at the site. 
These observations gave insight into the tools used (physical and digital), the robustness of the environment and the organization needed for the \autists[]. 

\subsubsection{Impressions}
\egebakken{} is a robust environment, in many ways resembling public schools, though with a greater focus on serving the needs of each \autist[]. 
The physical tools used at \egebakken{} fulfill simple tasks, and are robust, but also come with limitations. 
While they can be very flexible, they can require a sizable effort to work with. 
Some of these tools are suitable for having digital replacements, as proper software could lessen the previous issue. 
Software solutions do however already exist, but our customers find them hard to use and overly expensive. \todo{Find reelle tal.}
These solutions also come with limited flexibility, hampering the work of the \guardian[]s, as they try to optimize the solution to best enhance the understanding and communication skills of each \autist[]. 
The \guardian[]s are also in charge of organizing the daily schedule of each \autist[], with the child having some limited influence. 

\todo{Mangler TAIL}
