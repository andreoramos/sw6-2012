\section{Working Process}
The working process of this project is based on agile development methods, mainly Scrum and XP. 
There are two work forms used by us in this project, the one used by the entire \globalgroup{} and the one used by our \localgroup{}. 

\subsection{Multiproject work form}
The work form of the \globalgroup{} is designed around Scrum of Scrums. 
With it, the work for all the \localgroup{}s is split into sprints, and each \localgroup{} works in synchronization with the other \localgroup{}s. 
Each sprint is started with a meeting, where each \localgroup{} presents their progress so far and what they are going to accomplish in the coming sprint. 
These meetings are also used to vote on the length of the coming sprint, knowledge sharing and decision making for decisions that affect the entire \globalgroup{}. \newline
Each sprint is ended with an evaluation meeting, used to briefly sum up and evaluate on the sprint, such that each \localgroup{} recieves feedback on how they might improve their next sprint, before that sprint is planned. 
When two or more \localgroup{}s need to work on something not related to the \globalgroup{}, no formal meeting is required, and open discussion amongst the \localgroup{}s is encouraged through an open door policy. 
This has been important, as many \localgroup{}s rely on each other to provide services. 
This has also led to continuous, informal integration testing of the system, as each \localgroup{} increased their reliance on the others. 
\todo{Flet sammen med common report. Meget af det der staar her burde egentlig staa i common report}

\subsection{Launcher work form}
\todo{Find god overskrift}
The work form used in our \localgroup{} is primarily based on XP. 
XP is built around having a customer available at all times, but this has not been the case for this project. 
We found it unnecessary to compensate for this however. 
This was due to the nature of the project, which, in the case of the launcher, was focused a lot on usability and GUI, subjects that only require occasional user feedback to develop properly.
\todo{Hvorfor kraever det kun occasional feedback?}
The GUI focus also made it easier to fill out the backlog without the customer, and planning poker; a Scrum practice, which was used to determine the size of each task. \newline
Other XP conventions did make it into the work form though, e.g. pair programming. 
Pair programming has been a great tool in keeping the development pace up, as assisting each other in this manner makes it easier to discover problems and solutions early, while also reducing overhead in communicating code to the rest of the team. \newline
Another XP convention that helped reduce this overhead was refactoring. 
This involves going through existing code to rewrite parts that are complex from a readability point of view, in order to simplify the code and make it easier to understand. 
This is essential, as the project will be handed over to a new team later on, and high readability helps ensure that the project is useful to them. \newline
Lastly, the \textquotedblleft{}whole team\textquotedblright{} and \textquotedblleft{}sustainable pace\textquotedblright{} conventions were used as well. 
The reason for this was to ensure that our work remained high in quality, and led to fairly stringent work rules, where the team agreed to work through the day together, but not work at home. 
Exceptions were made in case of illness. \newline
One final note is that \textquotedblleft{}collective code ownership\textquotedblright{} was also employed. 
However, this is a demand from the study regulation, and not something that was deployed based on personal judgement.

\todo{der mangler at skrives om TDD} 