\section{System Description}
\label{common:sec:sys_description}
GIRAF is a collection of applications, either fully or partially interdependent, for the Android platform, designed to be used by guardians and children. GIRAF consists of five projects with various degree of interaction. These projects are named Launcher, PARROT, WOMBAT, Oasis, and Savannah. Each of the groups have produced individual products, which are parts of a greater project, GIRAF. 

\textbf{Launcher} handles execution of GIRAF apps, and at the same time it provides safety features to ensure that a user that is not authorized to interact with the rest of the system will not be able to do so. When the launcher executes an app, it will provide it with profile information, specifying which child is currently using the app, as well as which guardian is signed in.

\textbf{PARROT} is an app which provides access to pictograms -- pictures with associated information such as sound and text -- which can be used for communication. PARROT also gives guardians functionality for adding additional pictograms, as well as organizing the pictograms into categories for ease of access, based on the needs of the individual child.

\textbf{WOMBAT} is an app which purpose is to help the children to understand the aspect of time, by visualizing it. WOMBAT provides different ways of displaying time, as well as the possibility to configure the app for the needs of individual children. 

\textbf{Oasis} locally stores the data and configuration of the GIRAF platform, and provides an API to access it. The stored data and configurations are synchronized to the Savannah server, if available. In addition, an app is provided for the guardian to access the stored data and configurations.

\textbf{Savannah} provides Oasis with a way to synchronize tablets running GIRAF. Furthermore, a website is provided to ease administration of the synchronized data.