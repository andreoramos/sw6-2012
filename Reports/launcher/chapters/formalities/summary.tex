\pdfbookmark[1]{Summary}{Summary} % Bookmark name visible in a PDF viewer

\begingroup
\let\clearpage\relax
\let\cleardoublepage\relax
\let\cleardoublepage\relax

\chapter*{Summary}
\label{report_structure}

Institutions for children with ASD currently employ analogue tools which are suitable for digitization.
Such digitized tools exist today, but are however not cost effective for the institutions. 

This report describes the development of such a tool to help children with ASD and their guardians in their daily routines.
The solution is a platform for Android, called GIRAF, built for tablets.

GIRAF was developed using an agile work method by a multi-project group of 18 students. The multi-project consisted of five smaller groups, with each group responsible for an individual part of the system.
Five guardians, consisting of educators and pedagogues, were attached as customers, and each group had one customer assigned to them. \\

GIRAF consists of five parts. The first group developed an app that visualizes time. The second developed an app that enables communicating through images.
The third an administration module, that all GIRAF apps use in order to save their data.

The fourth group developed a server module, designed to allow data to be shared across devices.
The fifth group developed an app that allows the user to run GIRAF apps, which is the part that this report documents. \\

Through analysis, usability was found the most important quality for the launcher, and therefore, the highest priority while developing the launcher, was to ensure high usability. \\

This, among other design decisions, lead to a shared library being designed and implemented, such that common UI components were available for all parts of the GIRAF system. \\

Lastly, usability tests were performed, to reveal usability issues in the product.
The test revealed a single critical, three serious, and three cosmetic issues.
We believe these provide a good foundation for further development.

\endgroup			
\vfill
