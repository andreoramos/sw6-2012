\subsection{Authentication}
\label{backlog:authentication}

There was implemented a Authentication screen where the user with their QR code could login. The tablet will give a gesture, a short vibration, when the QR code is recognized.

\subsubsection{Autologin}
\label{backlog:autologin}
The autologin feature enables guardians to start the launcher without authentication, if authentication was performed within the last eight hours.

\begin{comment}
The autologin feature enables guardians to start the launcher without authentication, if authentication was performed within the last eight hours. This is to allow guardians to easily switch between different Android launchers, without having to re-authenticate every time the GIRAF launcher is accessed. Although not an highly important feature, this feature improved the development speed in such a degree that it simply was worth spending the time implementing it.

One could argue that this feature introduces a security issue, as an attacker, if able to physically get a locked device, running the GIRAF launcher in an authenticated session, could simply reboot the device in order to get full access.

We deem that this is not critical, since a simple workaround is to always logout whenever the device is placed in an location where unauthorized users might get a hold of it.
\end{comment}


\subsubsection{Authentication instructions and illustrations}
\label{backlog:authentication_illustrations}

There was made and implemented instrutions and illustrations so the user would easier understand the process of scanning an QR code.

\subsubsection{QR scanner}
\label{backlog:QR_scanner}

There was implemented a QR scanner using ZXing.