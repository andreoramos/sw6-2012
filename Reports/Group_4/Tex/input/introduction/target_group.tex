\section{Target Group}
Our target group is both children and their guardians. These guardians have certain needs for special tools and gadgets that help to ease the communication between them and the children.

Five teachers and educators, who work with children, act as customers. They will provide requirements and information about the institutions' way of working to give us an insight into their daily struggles.

\subsection{Working with Children with ASD}
This section is based upon the statements of a woman with ASD \cite{autism.com}, explaining what it is like to live with ASD, and an interview with an educator at Birken, a special kindergarten for children (see \autoref{InterviewMette} for interview notes).

	People with ASD are often more visual in their way of thinking. Rather than visualizing thoughts in language and text, they do it in pictures or visual demonstrations. Pictures and symbols are therefore an essential part of the daily tools used by children and the people interacting with them. Also, children can have difficulties expressing themselves by writing or talking, and can often more easily use electronic devices to either type a sentence or show pictures, to communicate with people around them.
	Another characteristic of children is their perception of time. Some of them simply do not understand phrases like ``in a moment'' or ``soon'', they will need some kind of visual indicator that shows how long time they will have to wait.

Different communication tools for children with autism already exist, but many of them rely on a static database of pictures, and often these has to be printed on paper in order to use them as intended. Other tools, such as hourglasses of different sizes and colors, are also essential when working with children, and these tools are either brought around with the child, or a set is kept every place the child might go, e.g. being at an institution or at home.

There exists tools today which helps the guardians in their daily life, although -- as stated in Drazenko's quote -- none of them are cost-effective enough to be used throughout the institutions. From the quote, it is clear that there is a need for a more cost-effective solution.

\begin{quotation}
\textit{``The price of the existing solutions are not sufficiently low such that we can afford to buy and use them throughout the institution.''}\\ 
	\begin{flushright}
		- Drazenko Banjak, educator at Egebakken.
	\end{flushright}
\end{quotation}