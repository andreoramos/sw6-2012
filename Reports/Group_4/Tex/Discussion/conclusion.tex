\chapter{Conclusion}
In this chapter we described the things we concluded along the project. 

\section{GIRAF Problem Definition}
In the multi project we made a problem definition. The problem definition is as follows:

\begin{quotation}
\textit{How can we ease the daily life for children with ASD and their guardians, while complying with the study regulation?}
\end{quotation}

To comply with the study regulations, we (the multi project) have designed and implemented a system called GIRAF.
To ensure that every project group has the opportunity to be up to date of the progress of the multi project, we (the multi project) agreed on using the same development method \vref{sec:dev_meth}.

\section{Oasis Problem Definition}
In Oasis we have specified the multi project problem definition to fit our project. 
The problem definition is a follows:

\begin{quotation}
	\textit{How can we provide a set of tools, which can help develop applications for the GIRAF system?}
\end{quotation}

We have designed and implemented an administration module for the GIRAF system.
The administration module consists of three parts; a local database, called Oasis Local Db, a library, called Oasis Lib, and an administration application, called Oasis App.
The Oasis Local Db ensures that the data is saved correctly. The Oasis Lib ensures that the different applications of the GIRAF system can interact with Oasis Local Db. The Oasis App ensures that the guardians can manage profiles of the GIRAF system, directly on the tablets.

\section{Oasis}
We began by examining the previous student reports, to check if there was any aspects we could reuse in the project.
After that we examined the possibilities of how to save data on an Android device, which lead us to start working on the architecture of the local database.
Along with that we gathered requirements from the other groups to start working on the architecture of the library.
When we finished the Oasis Local Db and the Oasis Lib we started working on the Oasis App.
The Oasis App shows some of the capabilities of the Oasis Lib, and by the same time give the guardians a possibility on managing the different profiles of the GIRAF system.

\section{Testing}
To verify the quality of the multi project, we conducted a usability test.
The test subjects consisted of the customers of the multi project.
The test highlighted some issues in the Oasis App, which could be corrected by the next group of developers, the issues can be seen in Section \vref{sec:usability_results}.
We created unit tests for the Oasis Lib.
The tests were made at the end of the project period.
This should be an ongoing process instead of doing them at the end of the semester.
The tests ensures indirectly, that the Oasis Local Db works.