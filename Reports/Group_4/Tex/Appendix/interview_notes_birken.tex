\section{Notes from Interview}
\label{InterviewMette}
\textit{This is notes from an interview with Mette Als Andreasen, an educator at Birken in Langholt, Denmark.}

N\aa{}r tiden l\o{}ber ud (kristian har tage et billede):\\
F\ae{}rdig - symbol\\
G\aa{} til skema - symbol\\
Taget fra boardmaker\\

Kunne v\ae{}re godt hvis man kunne s\ae{}tte egne billeder ind som start/stop symboler.\\


R\o{}d farve $=$ nej, stop, aflyst.\\

De har s\aa{}dan et ur p\aa{} 60 minutter hvor tid tilbage er markeret med r\o{}d, og s\aa{} bipper den lige kort n\aa{}r den er f\ae{}rdig.\\
  Det ville v\ae{}re fint hvis de kunne bruge sort/hvid til dem der ikke kan h\aa{}ndtere farver, men ogs\aa{} kan v\ae{}lge farver.\\

Stop-ur:\\
en fast timer p\aa{} 60 minutter $+$ en customizable som ikke ser helt magen til ud, som f.eks, kan v\ae{}re p\aa{} 5, 10 eller 15 minutter for en hel cirkel.\\

timeglas:\\
skift farve p\aa{} timeglassene, men ikke n\o{}dvendigvis g\o{}re dem st\o{}rre. Kombinere med mere/mindre sand. Eventuelt kombinere med et lille digitalt ur, til dem der har brug for det, skal kunne sl\aa{}es til og fra.\\

Dags-plan:\\
ikke s\ae{}rlig relevant til de helt sm\aa{} og ikke s\ae{}rligt velfungerende b\o{}rn. Men kunne v\ae{}re rigtig godt til de lidt \ae{}ldre.\\
   En plan g\aa{}r oppefra og ned, og hvis der s\aa{} skal specificeres noget ud til aktiviteterne, s\aa{} er det fra venstre mod h\o{}jre ud fra det nedadg\aa{}ende skema.\\

Til parrot:\\
Godt med rigtige billeder af tingene, som p\ae{}dagogerne selv kan tage, eventuelt ogs\aa{} af aktiviteter, s\aa{} pedagogerne kan have billeder af aktiviter som de kan liste efter skeamet.\\

Der var mange skemaer rundt omkring, og der henviser det sidste billede i r\ae{}kken til n\ae{}ste skema, som h\ae{}nger f.eks. p\aa{} badev\ae{}relset eller i garderoben.