\section{Briefing}
Goddag og velkommen til denne brugervenlighedsunders\o{}gelse.

Vi vil gerne starte med at takke dig for, at du vil hj\ae{}lpe os med at gennemf\o{}re denne brugervenlighedsunders\o{}gelse. Vi l\ae{}ser op fra dette dokument for at sikre os, at alle personer som deltager i vores studie for samme introduktion. Hvis du har sp\o{}rgsm\aa{}l undervejs, er du naturligvis meget velkommen til at stille disse sp\o{}rgsm\aa{}l.

Vi har i dette semester bygget et system til Android til at hj\ae{}lpe b\o{}rn med autisme og deres p\ae{}dagoger og for\ae{}ldre, og det er nu n\aa{}et til et stadie hvor vi gerne vil teste systemet. Denne test handler udelukkende om at finde problemer og mangler i systemet, og ikke om at teste jeres viden af systemet, s\aa{} alle tanker I m\aa{} have om produktet vil vi meget gerne h\o{}re.

F\o{}r vi starter f\o{}rste del af testen, vil jeg bede dig om at underskrive denne samtykkeerkl\ae{}ring for at sikre, at du er indforst\aa{}et med rammerne for studiet. Derudover skal du ogs\aa{} svare p\aa{} et demografisk sp\o{}rgeskema inden testen g\aa{}r i gang.

Testen best\aa{}r af fire dele:
\begin{itemize}
	\item Test af applikationer (20 min)
	\item De-briefing og sp\o{}rgeskema (5 min)
	\item Test af Administrations applikation og web applikation (20 min)
	\item De-briefing og sp\o{}rgeskema (5 min)
\end{itemize}

Undervejs vil der v\ae{}re en pause.

I de to tests vil du blive stillet en r\ae{}kke opgaver som du skal l\o{}se. L\ae{}s opgaveformuleringen grundigt og fort\ae{}l s\aa{} test hj\ae{}lperen hvad du mener opgaven g\aa{}r ud p\aa{}. Derefter skal du fors\o{}ge at l\o{}se opgaven s\aa{} godt som muligt. Opgaverne skal l\o{}ses i den r\ae{}kkef\o{}lge de st\aa{}r s\aa{}ledes at du starter med opgave 1 og arbejder dig ned af. 

Det er meningen at du skal t\ae{}nke h\o{}jt mens du l\o{}ser opgaverne. Dvs. at du siger hvad du har t\ae{}nkt dig at g\o{}re for at l\o{}se opgaven, hvilke ting du synes virker uklare eller komplicerede og hvordan du tror systemet virker. For eksempel vil det v\ae{}re godt hvis du n\ae{}vner hvad du forventer en knap g\o{}r inden du trykker p\aa{} den.

N\aa{}r testen er f\ae{}rdig vil der v\ae{}re nogle afsluttende sp\o{}rgsm\aa{}l som du skal besvare omkring hvordan du synes testen er forl\o{}bet og hvad din opfattelse af systemet er.