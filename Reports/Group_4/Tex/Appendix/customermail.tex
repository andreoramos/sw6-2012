\section{Mail correspondence with Customer}
\subsection{Mail To Customer}
Hej Kristine

Vi er blevet tildelt dig som kontakt person i forbindelse med vores projekt. Som n\ae{}vnt sidst s\aa{} arbejder vi p\aa{} at udvikle applikationer til android, som kan bruges enten af jer som p\ae{}dagoger og m\aa{}ske af autisterne p\aa{} sigt. Vi vil udvikle flere forskellige applikationer, og vi vil gerne l\o{}bende aftale m\o{}der med dig, hvor vi kan vise det samlede produkt som er lavet. P\aa{} den m\aa{}de kan vi f\aa{} feedback p\aa{} hvad der g\aa{}r godt og hvad der er knap s\aa{} godt.

Vores udviklings gruppe best\aa{}r af tre personer og vi skal lave en applikation der kan hj\ae{}lpe med at lave profiler der passer til b\o{}rnene. 

Da vi stadig kun er i gang med at planl\ae{}gge mener vi ikke at det er n\o{}dvendigt at holde et m\o{}de endnu. Men vi har nogen sp\o{}rgsm\aa{}l som vi gerne vil have dig til at svare p\aa{}:

\subsubsection{Hvilke informationer gemmer i omkring det enkelte barn?}
\begin{itemize}
	\item Journal nummer?
	\item Person nummer?
	\item Navn?
	\item Alder?
	\item S\ae{}rlige behov?
\end{itemize}

\subsubsection{a - Ur}
a1. Vil barnet kunne forst\aa{} at en hel cirkel kan have forskelligt tidsinterval?
a1.1 Eller er det bedst hvis cirklen har et fast tidsrum fx 1 time?
\\
a2. Hvis man skal m\aa{}le et tidsinterval p\aa{} uret, er det s\aa{} bedst at lade uret efterligne et almindeligt ur med 12 timer eller et stop-ur med kun 1 time?

\subsubsection{b - Timeglas}
b1. Vil barnet kunne forst\aa{} at det samme timeglas med den samme m\ae{}ngde sand kan varierer i tid?
\\
b2. Er det bedst at man varierer i m\ae{}ngden af sand i timeglasset eller at man varierer i timeglassets st\o{}rrelse?

\subsubsection{c - Aktivitetstid}
c1. Vil barnet kunne forst\aa{} at en linje der g\aa{}r hele vejen hen over sk\ae{}rmen kan varierer i tidsinterval?      
c1.1 Eller er det bedre hvis linjen har et fast tidsinterval og fx en halv linje derfor svarer til en halv time og en hel linje til en hel time? 

\subsubsection{d - Dagsplan}
d1. Hvis man laver en visuel dagsplan er det s\aa{} bedst at man laver et interval som viser tiden imellem to aktiviteter, eller at man viser alle aktiviter i l\o{}bet af dagen kombineret med en tidslinje?

\subsection{Mail From Customer}
Hej.
Tak for jeres mail.
Jeg skal besvare jeres mail s\aa{} godt som muligt, og s\aa{} m\aa{} i give lys hvis i har brug for at jeg uddyber.
\\
Vedr. informationer vedr. barnet:
Vi benytter et elektronisksystem som hedder, EKJ, hvor alle oplysninger p\aa{} b\o{}rnene er gemt. Det vil sige, pers. nr., adresse oplysninger, indbydelser, handleplaner og referater fra diverse m\o{}der.
 \\
UR: Hvis det er tydeligt vist at ``tiden g\aa{}r'' /skiven bliver mindre/forsvinder, som tiden g\aa{}r, vil barnet forst\aa{} meningen med uret. For at indikere forskellig tid, kan man benytte forskellige farvet baggrunde. Lilla:5 min. Gr\o{}n:10 min osv. Vi benytter kun kortere tidsintervaller,(1. min. 3. min. 5 min. -op til ca. 10-15. min) da 1 time er for abstrakt.
\\
Timeglas: Hvis der er en tydelig markering af tidsintervallet, som beskrevet ovenfor, er det muligt at bruge samme timeglas. Tror det vil give bedst forst\aa{}else for barnet, hvis m\ae{}ngden af sand varieres efter tid. 
\\
Aktivitetstid og dagsplan: (Tror J) Aktivitetstid kan bruges ved, at tiden bliver indikeret af m\ae{}ngden af aktiviteter.
 - Alts\aa{} 3-5 viste aktiviteter af gangen, og ikke s\aa{} meget om det er en time eller 15 min. 
Tiden kunne v\ae{}re en mulighed at tilf\o{}re, om n\o{}dvendigt. Mange af vores b\o{}rn har manglende fornemmelse for tid, og ofte har de brug for at se sm\aa{} konkrete sekvenser/beskeder frem for mange over l\ae{}ngere tid.
Derfor vil jeg tror de bedst kan overskue \textonehalf{} dag af gangen, men stadig have mulighed for at have dagen p\aa{} skemaet, hvor det kan vises i sekvenser.
\\
Jeg har samlet de to ovenst\aa{}ende punkter, da de nemt kommer til at gribe ind i hinanden.
Vores ugeskemaer i b\o{}rnehaven er vist med internationale farver, dem vil i ligeledes kunne benytte til at tydeligg\o{}re ugedagene.
Mandag: Gr\o{}n, Tirs.: Lilla, Ons.: orange, tors.: bl\aa{}, Fre.: gul, l\o{}r.: r\o{}d og s\o{}ndag: hvid.
\\
H\aa{}ber dette er uddybende nok, ellers m\aa{} i gerne skrive eller ringe til mig hvis det er nemmere.
\\
Ser  frem til at h\o{}rer fra jer igen.